\documentclass[../../sp_2016.tex]{subfiles}

\graphicspath{{./image/}}

\begin{document}
\section{力学: 2重振子}
\subsection{}
\begin{equation}\begin{aligned}[b]
    L = \frac{m_1}{2}(\dot{x}_1^2\dot{y}_1^2)+\frac{m_2}{2}(\dot{x}_2^2\dot{y}_2^2)+m_1gy_1+m_2gy_2
\end{aligned}\end{equation}
\subsection{}
\begin{equation}\begin{aligned}[b]
    x_1 &= l_1\sin\theta_1 & y_1 &= l_1\cos\theta\\
    x_2 &= l_1\sin\theta_1+l_2\sin\theta_2 &  y_2&= l_1\cos\theta_1+l_2\cos\theta_2
\end{aligned}\end{equation}

\subsection{}
\begin{equation}\begin{aligned}[b]
    \dot{x}_1 &= l_1\dot{\theta}_1\cos\theta_1 & y_1 &= -l_1\dot{\theta}_1\sin\theta\\
    \dot{x}_2&= l_1\dot{\theta}_1\cos\theta_1+l_2\dot{\theta}_2\cos\theta_2 &
    \dot{y}_2&= -l_1\dot{\theta}_1\sin\theta_1-l_2\dot{\theta}_2\sin\theta_2
\end{aligned}\end{equation}
であるのでラグランジアンは適宜三角関数のTaylor展開を使って、
\begin{equation}\begin{aligned}[b]
    L &= \frac{m_1l_1^2}{2}\dot{\theta}_1^2+\frac{m_2}{2}\Bigl[
        l_1^2\dot{\theta}_1^2+l_2^2\dot{\theta}_2^2+2l_1l_2\dot{\theta}_1\dot{\theta}_2\cos(\theta_1-\theta_2)
    \Bigr]+mgl_1\cos\theta_1+m_2gl_1\cos\theta_1+m_2gl_2\cos\theta_2\\
    &\simeq
    \frac{1}{2}(m_1+m_2)l_1^2\dot{\theta}_1^2+\frac{1}{2}m_2l_2^2\dot{\theta}_2^2+m_2l_1l_2\dot{\theta}_1\dot{\theta}_2
    -\frac{1}{2}(m_1+m_2)gl_1\theta_1-\frac{1}{2}m_2gl_2\theta_2^2+\text{const.}
\end{aligned}\end{equation}

\subsection{}
オイラーラグランジュ方程式方程式より\(\theta_1\)から得られる微分方程式は
\begin{equation}\begin{aligned}[b]
    \dv{t} \pdv{L}{\dot{\theta}_1} &= \pdv{L}{\theta_1}\\
    (m_1+m_2)l_1\ddot{\theta}_1+m_2l_2\ddot{\theta} &=-(m_1+m_2)g\theta_1,
\end{aligned}\end{equation}
\(\theta_2\)から得られる微分方程式は
\begin{equation}\begin{aligned}[b]
    \dv{t} \pdv{L}{\dot{\theta}_2} &= \pdv{L}{\theta_2}\\
    l_1\ddot{\theta}_1+l_2\ddot{\theta} &=-g\theta_2
\end{aligned}\end{equation}
となる。これに\(\theta_1(t)=a_1\cos\omega t,\theta_2(t)=a_2\cos\omega t\)を代入すると
\begin{equation}\begin{aligned}[b]
    &\begin{cases}
        (m_1+m_2)(g-l_1\omega^2)a_1-m_2l_2\omega^2 a_2 &=0\\
        -l_1\omega^2 a_1 +(g-l_2\omega^2)a_2 &= 0
    \end{cases}\\
    &\begin{pmatrix}
        (m_1+m_2)(g-l_1 \omega^2) & -m_2l_2\omega^2\\
        -l_1\omega^2 &(g-l_2\omega^2)
    \end{pmatrix}\begin{pmatrix}
        a_1\\a_2
    \end{pmatrix}=\begin{pmatrix}
        0 \\ 0
    \end{pmatrix}
\end{aligned}\end{equation}
これより、求める固有角振動数はこの連立方程式が非自明な解をもつような\(\omega\)のことである。
行列式の値が0のとき、非自明な解となるので
\begin{equation}\begin{aligned}[b]
    0 &= (m_1+m_2)(g-l_1\omega^2)(g-l_2\omega^2)-m_2l_1l_2\omega^4\\
    &=m_1l_1l_2\omega^4-g(m_1+m_2)(l_1+l_2)\omega^2+(m_1+m_2)g^2\\
    \omega^2 &= \frac{g(m_1+m_2)(l_1+l_2)\pm\sqrt{g^2(m_1+m_2)^2(l_1+l_2)^2-4m_1(m_1+m_2)l_1l_2g^2}}{2m_1l_1l_2}\\
    \omega &= \sqrt{\frac{(1+m_2/m_1)g(l_1+l_2)\pm g\sqrt{(1+m_2/m_1)^2(l_1+l_2)^2-4(1+m_2/m_1)l_1l_2}}{2l_1l_2}}
\end{aligned}\end{equation}

\subsection{}
\(m_1\gg m_2\)のとき固有振動数は
\begin{equation}\begin{aligned}[b]
    \omega^2 &= \frac{(l_1+l_2)g\pm g\sqrt{(l_1+l_2)^2-4l_1l_2}}{2l_1l_2}\\
    &= \frac{g}{2l_1l_2}\Bigl[(l_1+l_2)\pm\abs{l_1-l_2}\Bigr]\\
    &= \frac{g}{l_1},\,\frac{g}{l_2}
\end{aligned}\end{equation}
とわかる。

固有振動数\(\omega_1^2\equiv g/l_1\)のとき、設問4の行列に代入すると\(m_2\ll 1\)に注意すると
\begin{equation}\begin{aligned}[b]
    \begin{pmatrix}
        0 & 0 \\
        -g & g-gl_2/l_1
    \end{pmatrix}\begin{pmatrix}
        a_1\\a_2
    \end{pmatrix}
    &=\begin{pmatrix}
        0\\ 0
    \end{pmatrix}\\
    \frac{a_1}{a_2} &= 1-\frac{l_2}{l_1}
\end{aligned}\end{equation}
となる。

固有振動数\(\omega_2^2\equiv g/l_2\)のとき、設問4の行列に代入すると
\begin{equation}\begin{aligned}[b]
    \begin{pmatrix}
        g-gl_1/l_2 & 0 \\
        -gl_1/l_2 & 0
    \end{pmatrix}\begin{pmatrix}
        a_1\\a_2
    \end{pmatrix}
    &=\begin{pmatrix}
        0\\ 0
    \end{pmatrix}\\
    a_1 &= 0
\end{aligned}\end{equation}
となる。

以上より
\begin{equation}\begin{aligned}[b]
    (a_1,a_2) = (l_1-l_2,l_2)
\end{aligned}\end{equation}
の振幅比で振動数が\(\omega^2=g/l_1\)のモードと
\begin{equation}\begin{aligned}[b]
    (a_1,a_2) = (0, 1)
\end{aligned}\end{equation}
の振幅比で振動数が\(\omega^2=g/l_2\)のモードの2つがある。

\subsection{}
わかんない


\subsection*{感想}
ニュートン力学と比べるとこっちのが楽だけど、
それでも依然面倒ではある。

\end{document}