\documentclass[../../sp_2019.tex]{subfiles}

\graphicspath{{./image/}}

\begin{document}
\setcounter{section}{0}
\section{量子力学: エンタングルメント・局所実在性}
測定結果がとりうる状態というのを、
一般の状態にたいして、観測量がとりうる値つまり、観測量の固有値と解釈する。
\subsection{}
\(\sigma_z\)の固有値固有状態のペアは
\begin{equation}\begin{aligned}[b]
    \qty(s_z=+1;\,\ket{\uparrow}),
    \qty(s_z=-1;\,\ket{\downarrow})
\end{aligned}\end{equation}
である。\(s_z\)の期待値は
\begin{equation}\begin{aligned}[b]
    \ev{s_z} &= +1\times P(\uparrow)-1\times P(\downarrow) \\
    &= \abs{\braket{\uparrow}{\uparrow}}^2-\abs{\braket{\downarrow}{\uparrow}}^2\\
    &= 1
\end{aligned}\end{equation}

\subsection{}
\(\sigma_x\)の固有値固有状態のペアは
\begin{equation}\begin{aligned}[b]
    \qty(s_x=+1;\,\ket{+}\equiv \frac{\ket{\uparrow}+\ket{\downarrow}}{\sqrt{2}}),
    \qty(s_x=-1;\,\ket{-}\equiv \frac{\ket{\uparrow}-\ket{\downarrow}}{\sqrt{2}})
\end{aligned}\end{equation}
である。\(s_x\)の期待値は
\begin{equation}\begin{aligned}[b]
    \ev{s_x} &= +1\times P(+)-1\times P(-) \\
    &= \abs{\braket{+}{\uparrow}}^2-\abs{\braket{-}{\uparrow}}^2\\
    &= \frac{1}{2}-\frac{1}{2} =0
\end{aligned}\end{equation}

\subsection{}
\(\sigma(\theta)\)の固有値固有状態のペアは
\begin{equation}\begin{aligned}[b]
    \qty(s_\theta = +1;\,\ket{+\theta}\equiv \cos\frac{\theta}{2}\ket{\uparrow}+\sin\frac{\theta}{2}\ket{\downarrow}),
    \qty(s_\theta = -1;\,\ket{-\theta}\equiv \sin\frac{\theta}{2}\ket{\uparrow}-\cos\frac{\theta}{2}\ket{\downarrow})
\end{aligned}\end{equation}
である。\(s_x\)の期待値は
\begin{equation}\begin{aligned}[b]
    \ev{s_x} &= +1\times P(+\theta)-1\times P(-\theta) \\
    &= \abs{\braket{+\theta}{\uparrow}}^2-\abs{\braket{-\theta}{\uparrow}}^2\\
    &= \cos^2\frac{\theta}{2}-\sin^2\frac{\theta}{2} =\cos\theta
\end{aligned}\end{equation}

\subsection{}
\(\sigma_z^A\sigma_z^B\)の固有値固有状態のペアは
\begin{equation}\begin{aligned}[b]
    &\qty(s_z^A = +1,s_z^B = +1;\,\ket{\uparrow\uparrow} \equiv \ket{\uparrow}_A\ket{\uparrow}_B)\\
    &\qty(s_z^A = +1,s_z^B = -1;\,\ket{\uparrow\downarrow} \equiv \ket{\uparrow}_A\ket{\downarrow}_B)\\
    &\qty(s_z^A = -1,s_z^B = +1;\,\ket{\downarrow\uparrow} \equiv \ket{\downarrow}_A\ket{\uparrow}_B)\\
    &\qty(s_z^A = -1,s_z^B = -1;\,\ket{\downarrow\downarrow} \equiv \ket{\downarrow}_A\ket{\downarrow}_B)
\end{aligned}\end{equation}
である。\(s_z^As_z^B\)の期待値は
\begin{equation}\begin{aligned}[b]
    \ev{s_z^As_z^B}
    &= +1\times P(\uparrow\uparrow) -1\times P(\uparrow\downarrow)
        -1\times P(\downarrow\uparrow) +1\times P(\downarrow\downarrow)\\
    &= \abs{\braket{\uparrow\uparrow}{\Psi}}^2 -\abs{\braket{\uparrow\downarrow}{\Psi}}^2
        -\abs{\braket{\downarrow\uparrow}{\Psi}}^2+\abs{\braket{\downarrow\downarrow}{\Psi}}^2\\
    &= 0 -\frac{1}{2} -\frac{1}{2} +0 = -1
\end{aligned}\end{equation}

\subsection{}
\(\sigma_x^A\sigma_x^B\)の固有値固有状態のペアは
\begin{equation}\begin{aligned}[b]
    &\qty(s_x^A = +1,s_x^B = +1;\,\ket{++} \equiv \ket{+}_A\ket{+}_B=\frac{\ket{\uparrow\uparrow}+\ket{\uparrow\downarrow}+\ket{\downarrow\uparrow}+\ket{\downarrow\downarrow}}{2})\\
    &\qty(s_x^A = +1,s_x^B = -1;\,\ket{+-} \equiv \ket{+}_A\ket{-}_B=\frac{\ket{\uparrow\uparrow}-\ket{\uparrow\downarrow}+\ket{\downarrow\uparrow}-\ket{\downarrow\downarrow}}{2})\\
    &\qty(s_x^A = -1,s_x^B = +1;\,\ket{-+} \equiv \ket{-}_A\ket{+}_B=\frac{\ket{\uparrow\uparrow}+\ket{\uparrow\downarrow}-\ket{\downarrow\uparrow}-\ket{\downarrow\downarrow}}{2})\\
    &\qty(s_x^A = -1,s_x^B = -1;\,\ket{--} \equiv \ket{-}_A\ket{-}_B=\frac{\ket{\uparrow\uparrow}-\ket{\uparrow\downarrow}-\ket{\downarrow\uparrow}+\ket{\downarrow\downarrow}}{2})
\end{aligned}\end{equation}
である。\(s_x^As_x^B\)の期待値は
\begin{equation}\begin{aligned}[b]
    \ev{s_x^As_x^B}
    &= +1\times P(++) -1\times P(+-)
        -1\times P(-+) +1\times P(--)\\
    &= \abs{\braket{++}{\Psi}}^2 -\abs{\braket{+-}{\Psi}}^2
        -\abs{\braket{-+}{\Psi}}^2+\abs{\braket{--}{\Psi}}^2\\
    &= 0 -\frac{1}{2} -\frac{1}{2} +0 = -1
\end{aligned}\end{equation}

\subsection{}
\(\sigma_\theta^A\sigma_\theta^B\)の固有値固有状態のペアは
\begin{equation}\begin{aligned}[b]
    &\qty(s_\theta^A = +1,s_\theta^B = +1;\,\ket{+\theta+\theta} \equiv \ket{+\theta}_A\ket{+\theta}_B
        =\cos^2\frac{\theta}{2}\ket{\uparrow\uparrow}+\cos\frac{\theta}{2}\sin\frac{\theta}{2}\ket{\uparrow\downarrow}
        +\cos\frac{\theta}{2}\sin\frac{\theta}{2}\ket{\downarrow\uparrow}+\sin^2\frac{\theta}{2}\ket{\downarrow\downarrow})\\
    &\qty(s_\theta^A = +1,s_\theta^B = -1;\,\ket{+\theta-\theta} \equiv \ket{+\theta}_A\ket{-\theta}_B
        =\cos\frac{\theta}{2}\sin\frac{\theta}{2}\ket{\uparrow\uparrow}-\cos^2\frac{\theta}{2}\ket{\uparrow\downarrow}
        +\sin^2\frac{\theta}{2}\ket{\downarrow\uparrow}-\cos\frac{\theta}{2}\sin\frac{\theta}{2}\ket{\downarrow\downarrow})\\
    &\qty(s_\theta^A = -1,s_\theta^B = +1;\,\ket{-\theta+\theta} \equiv \ket{-\theta}_A\ket{+\theta}_B
        =\cos\frac{\theta}{2}\sin\frac{\theta}{2}\ket{\uparrow\uparrow}+\sin^2\frac{\theta}{2}\ket{\uparrow\downarrow}
        -\cos^2\frac{\theta}{2}\ket{\downarrow\uparrow}-\cos\frac{\theta}{2}\sin\frac{\theta}{2}\ket{\downarrow\downarrow})\\
    &\qty(s_\theta^A = -1,s_\theta^B = -1;\,\ket{-\theta-\theta} \equiv \ket{-\theta}_A\ket{-\theta}_B
        =\sin^2\frac{\theta}{2}\ket{\uparrow\uparrow}-\cos\frac{\theta}{2}\sin\frac{\theta}{2}\ket{\uparrow\downarrow}
        -\cos\frac{\theta}{2}\sin\frac{\theta}{2}\ket{\downarrow\uparrow}+\cos^2\frac{\theta}{2}\ket{\downarrow\downarrow})
\end{aligned}\end{equation}
である。初期状態\(\ket{\Psi}\)での確率は
\begin{equation}\begin{aligned}[b]
    P(+\theta+\theta) &= \abs{\braket{+\theta+\theta}{\Psi}}^2 = 0\\
    P(+\theta-\theta) &= \abs{\braket{+\theta-\theta}{\Psi}}^2 = \frac{1}{2}\\
    P(-\theta+\theta) &= \abs{\braket{-\theta+\theta}{\Psi}}^2 = \frac{1}{2}\\
    P(-\theta-\theta) &= \abs{\braket{-\theta-\theta}{\Psi}}^2 = 0
\end{aligned}\end{equation}
となる。
この結果は\(s_\theta^A=s_\theta^B\)になることはなく、
必ず\(s_\theta^A=-s_\theta^B=\pm 1\)になるということを表す。

\subsection{}
\(\sigma_\theta^A\sigma_\varphi^B\)の固有値固有状態のペアは
\begin{equation}\begin{aligned}[b]
    &\qty(s_\theta^A = +1,s_\varphi^B = +1;\,\ket{+\theta+\varphi} \equiv \ket{+\theta}_A\ket{+\varphi}_B
        =\cos\frac{\theta}{2}\cos\frac{\varphi}{2}\ket{\uparrow\uparrow}+\cos\frac{\theta}{2}\sin\frac{\varphi}{2}\ket{\uparrow\downarrow}
        +\sin\frac{\theta}{2}\cos\frac{\varphi}{2}\ket{\downarrow\uparrow}+\sin\frac{\theta}{2}\frac{{\varphi}}{2}\ket{\downarrow\downarrow})\\
    &\qty(s_\theta^A = +1,s_\varphi^B = -1;\,\ket{+\theta-\varphi} \equiv \ket{+\theta}_A\ket{-\varphi}_B
        =\cos\frac{\theta}{2}\sin\frac{\varphi}{2}\ket{\uparrow\uparrow}-\cos\frac{\theta}{2}\cos\frac{\varphi}{2}\ket{\uparrow\downarrow}
        +\sin\frac{\theta}{2}\sin\frac{\varphi}{2}\ket{\downarrow\uparrow}-\sin\frac{\theta}{2}\cos\frac{\varphi}{2}\ket{\downarrow\downarrow})\\
    &\qty(s_\theta^A = -1,s_\varphi^B = +1;\,\ket{-\theta+\varphi} \equiv \ket{-\theta}_A\ket{+\varphi}_B
        =\sin\frac{\theta}{2}\cos\frac{\varphi}{2}\ket{\uparrow\uparrow}+\sin\frac{\theta}{2}\sin\frac{\varphi}{2}\ket{\uparrow\downarrow}
        -\cos\frac{\theta}{2}\cos\frac{\varphi}{2}\ket{\downarrow\uparrow}-\cos\frac{\theta}{2}\sin\frac{\varphi}{2}\ket{\downarrow\downarrow})\\
    &\qty(s_\theta^A = -1,s_\varphi^B = -1;\,\ket{-\theta-\varphi} \equiv \ket{-\theta}_A\ket{-\varphi}_B
        =\sin\frac{\theta}{2}\sin\frac{\varphi}{2}\ket{\uparrow\uparrow}-\cos\frac{\theta}{2}\sin\frac{\varphi}{2}\ket{\uparrow\downarrow}
        -\sin\frac{\theta}{2}\cos\frac{\varphi}{2}\ket{\downarrow\uparrow}+\cos\frac{\theta}{2}\frac{\varphi}{2}\ket{\downarrow\downarrow})\\
\end{aligned}\end{equation}
である。\(s_\theta^As_\varphi^B\)の期待値は
\begin{equation}\begin{aligned}[b]
    \ev{s_x^As_x^B}
    &= +1\times P(+\theta+\varphi) -1\times P(+\theta-\varphi)
        -1\times P(-\theta+\varphi) +1\times P(-\theta-\varphi)\\
    &= \abs{\braket{+\theta+\varphi}{\Psi}}^2 -\abs{\braket{+\theta-\varphi}{\Psi}}^2
        -\abs{\braket{-\theta+\varphi}{\Psi}}^2+\abs{\braket{-\theta-\varphi}{\Psi}}^2\\
    &= \frac{1}{2}\sin^2(\theta-\varphi) -\frac{1}{2}\cos^2(\theta-\varphi) -\frac{1}{2}\cos^2(\theta-\varphi) + \frac{1}{2}\sin^2(\theta-\varphi) \\
    &= -\cos(2\theta-2\varphi)
\end{aligned}\end{equation}

\subsection{}
\(s^As^B\)の散りうる値は
\begin{equation}\begin{aligned}[b]
    (s^A,s^B)=(+1,+1),(+1,-1),(-1,+1),(-1,-1)
\end{aligned}\end{equation}
\(s^As^B\)の期待値を考える。
\(\theta\)と\(\varphi\)が同じ角度になる確率は\(1/3\)で期待値は\(-1\)である。
\(\theta\)と\(\varphi\)が違う角度になる確率は\(2/3\)で期待値は\(-\cos(\ang{120})=\frac{1}{2}\)である。
なので
\begin{equation}\begin{aligned}[b]
    \ev{s^As^B}  -1\times\frac{1}{3} + \frac{1}{2}\times \frac{2}{3} = 0
\end{aligned}\end{equation}

\subsection{}
\subsection*{(i)}
\(\theta\)と\(\varphi\)がどの方向でも\(s^As^B=-1\)なので、
\begin{equation}\begin{aligned}[b]
    \ev{s^As^B} = -1
\end{aligned}\end{equation}
\subsection*{(ii)}
A が \(\ang{0},\,\ang{120}\)、かつBが\(\ang{0},\ang{120}\)のとなる確率は\(4/9\)
A が \(\ang{240}\)、かつBが\(\ang{240}\)のとなる確率は\(1/9\)である。
なので\(s^As^B=-1\)となる確率は\(5/9\)。
残りが\(s^As^B=1\)となる状況で、その確率は\(4/9\).
なので
\begin{equation}\begin{aligned}[b]
    \ev{s^As^B} = -1\times \frac{5}{9}+1\times\frac{4}{9} = -\frac{1}{9}
\end{aligned}\end{equation}

\subsection*{(iii)}
考えるべき場合は一見\(8\)通りあるが、
\(s\)の値がすべて同じとき、
\(s\)の値が1つだけ違うときの2種類にわけれて、
このときの\(\ev{s^As^B}\)の値は既に(i)(ii)で調べた。
両者とも期待値の値は負であるので、古典的に扱ったときの\(s^As^B\)の期待値は
量子的に扱った場合と違って負の値となる。



\subsection*{感想}
量子情報をやったことない人には"測定した"の意味が分かりくく、
やるべきことを読み取れなかったら大門全部落としだし、
やったことある人にとっては簡単で同じ内容をただひたすら書くだけのただ手が疲れて時間がとられるというひどい問題。

量子情報が流行り始めたは2019年10月のGoogleによる実機の量子計算機での量子超越性が示されたというのがきっかけで、
この試験が行われた直後での話なので、本番でのできはひどそう。
これ解けた人はJ.J.SakuraiでのSG実験の問題をやっていたか、
上田先生まわりの量子光学をやってる人ぐらいな気がする。

この問題では量子力学の物理量は測定する前には決まっていないということを追っかけていたが、
部分系Aと部分系Bを十分離して片方だけ測定するという設定にしてこれをナイーブに考えると、
測定による相互作用が光速を超えるという結果が得られて相対論に反する結果となるのがわかる。
これを問題にするのは難しそう。



\end{document}
