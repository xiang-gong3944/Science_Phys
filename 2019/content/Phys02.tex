\documentclass[../../sp_2019.tex]{subfiles}

\graphicspath{{./image/}}

\begin{document}
\setcounter{section}{0}
\section{統計力学: 理想気体}
\subsection{}
(2)式にハミルトニアンを代入する。
分配関数は位置の積分を先にやって
\begin{equation}\begin{aligned}[b]
    Z(T,V,N)
    &= \frac{V^N}{N!h^{3N}}\qty(\int\dd{p_x}\dd{p_y}\dd{p_z}e^{-\frac{\beta\hbar^2}{2m}(p_x^2+p_y^2+p_z^2)})^N\\
    &= \frac{V^N}{N!}\qty(\frac{mk_BT}{2\pi\hbar^2})^{3N/2}
\end{aligned}\end{equation}
これより、自由エネルギーは
\begin{equation}\begin{aligned}[b]
    F &= -k_BT \ln Z \\
    &= -Nk_BT\ln\qty[\frac{V}{N}\qty(\frac{mk_BT}{2\pi\hbar^2})e]
\end{aligned}\end{equation}
よって圧力は
\begin{equation}\begin{aligned}[b]
    p = -\pdv{F}{V} = \frac{Nk_BT}{V}
\end{aligned}\end{equation}

\subsection{}
\(1/N!\)の因子がついていないときの分配関数\(Z'\)は
\begin{equation}\begin{aligned}[b]
    Z' = V^N \qty(\frac{mk_BT}{2\pi\hbar^2})^{3N/2}
\end{aligned}\end{equation}
より、
これから得られるヘルムホルツの自由エネルギー\(F'\)は
\begin{equation}\begin{aligned}[b]
    F' = -Nk_BT\ln\qty[V\qty(\frac{mk_BT}{2\pi\hbar^2})^{3/2}]
\end{aligned}\end{equation}
である。
ヘルムホルツの自由エネルギーは示量性を持つはずだが、
右辺の体積と粒子数を\(\lambda\)倍しても\(F'\)が\(\lambda\)倍になっていないことから、
示量性という熱力学的性質を満たしていないのがわかる。

\subsection{}
エントロピーは
\begin{equation}\begin{aligned}[b]
    S = -\pdv{F}{T} = Nk_B\ln\qty[\frac{V}{N}\qty(\frac{mk_BT}{2\pi\hbar^2})^{3/2}e] + \frac{3}{2}Nk_B
    = Nk_B\ln\qty[\frac{V}{N}\qty(\frac{mk_BT}{2\pi\hbar^2})^{3/2}e^{5/2}]
\end{aligned}\end{equation}
\(T\to0\)では\(S\to -\infty\)になり、
発散するため熱力学第3法則に反する結果となっている。

\subsection{}
\begin{equation}\begin{aligned}[b]
    \frac{C_V}{T} &= -\pdv[2]{F}{T} = -\pdv{S}{T}\\
    C_V &=  \frac{3}{2}Nk_B
\end{aligned}\end{equation}

\subsection{}
自由粒子の波動関数は
\begin{equation}\begin{aligned}[b]
    \psi(x,y,z)=e^{i(k_xx+k_yy+k_zz)}
\end{aligned}\end{equation}
のように書け、
周期境界条件
\begin{equation}\begin{aligned}[b]
    \psi(x+L,y,z)=\psi(x,y+L,z)=\psi(x,y,z+L)=\psi(x,y,z)
\end{aligned}\end{equation}
を満たさないといけないので波数は整数\(n_x,n_y,n_z\)を用いて
\begin{equation}\begin{aligned}[b]
    (k_x,k_y,k_z)= \frac{2\pi}{L}(n_x,n_y,n_z)
\end{aligned}\end{equation}
のように書ける。

\subsection{}
\begin{equation}\begin{aligned}[b]
    \bar{N} &= \frac{1}{\Xi}\prod_k\sum_{n_k=0,1} n_ke^{-\beta(\varepsilon_k-\mu)n_k}
    = \frac{1}{\Xi}\frac{\partial}{\beta\partial\mu}\prod_k(1+e^{-\beta(\varepsilon_k-\mu)})\\
    &= \frac{\partial}{\beta\partial\mu}\ln\Xi
    = \frac{\partial}{\beta\partial\mu}\sum_k\ln(1+e^{-\beta(\varepsilon_k-\mu)})
    = \sum_k \frac{1}{e^{\beta(\varepsilon_k-\mu)}+1} = \sum_k f(\varepsilon_k)
\end{aligned}\end{equation}

\subsection{}
\begin{equation}\begin{aligned}[b]
    N &= \sum_k e^{-\beta(\varepsilon_k-\mu)}
        = \frac{V}{(2\pi)^3}e^{\beta\mu}\int \dd[3]{k} e^{-\beta\varepsilon_k}\\
    &= \frac{V}{2\pi^2}e^{\beta\mu}\int_{0}^{\infty}k^2\dd{k}e^{-\frac{\beta \hbar^2}{2m}k^2}
        =\frac{V}{2\pi^2}\frac{\sqrt{\pi}}{4}\qty(\frac{2mk_BT}{\hbar^2})^{3/2}e^{\beta\mu}
        =V\qty(\frac{mk_BT}{2\pi\hbar^2})^{3/2}e^{\beta\mu}\\
    e^{-\beta\mu}&= \frac{V}{N}\qty(\frac{mk_BT}{2\pi\hbar^2})^{3/2}\\
    \mu &= -k_BT\ln\qty[\frac{V}{N}\qty(\frac{mk_BT}{2\pi\hbar^2})^{3/2}]
\end{aligned}\end{equation}

\subsection{}
エントロピーはグランドポテンシャル\(J=-k_BT\ln\Xi\)を使って\(S=-\partial J/\partial T\)と表せるので、
\begin{equation}\begin{aligned}[b]
    S &= \pdv{T}\sum_k k_BT\ln(1+e^{-\beta(\varepsilon_k-\mu)})\\
    & \simeq \pdv{T}\sum_k k_BTe^{-\beta(\varepsilon_k-\mu)}
        =\pdv{T}Nk_BT
        = k_BT\pdv{N}{T}+Nk_B
        = Nk_B\qty[\frac{5}{2}-\frac{\mu}{k_BT}]\\
    &\simeq -\frac{\mu}{T}N
\end{aligned}\end{equation}

\subsection{}
温度が高いときの比熱の温度特性は
\begin{equation}\begin{aligned}[b]
    C_V &= T\pdv{S}{T}
        = -\frac{\mu}{T}N -N\pdv{\mu}{T}
        = -\frac{\mu}{T}N +\frac{3}{2}Nk_B +\frac{\mu}{T}N
        =  \frac{3}{2}Nk_B
\end{aligned}\end{equation}
となり、古典的に扱ったときと同じ結果になる。

これとフェルミオンの比熱が低温では\(T\)に比例することと併せてグラフを書ける。

\subsection*{感想}
理想気体の典型問題。

最後、自由フェルミ気体の比熱の高温極限を調べるとき、\(\partial\mu/\partial T\)じゃなくて、
\(\partial N/\partial T\)でやってしまって変な答えになってしまった。


\end{document}
