\documentclass[../../master.tex]{subfiles}

\graphicspath{{./image/}}

\begin{document}
\section{量子力学: コヒーレント状態}
\subsection{}
\begin{align}
    \text{ア:} &\qquad \hat{a}^\dagger = \frac{1}{\sqrt{2}}\qty(\sqrt{\frac{m\omega}{\hbar}}\hat{x}-\frac{i}{\sqrt{m\hbar\omega}}\hat{p})\\
    \text{イ:} &\qquad \qty[\hat{a},\,\hat{a}^\dagger] = 1\\
    \text{ウ:} &\qquad \hat{x} = \sqrt{\frac{\hbar}{m\omega}}\frac{\hat{a}+\hat{a}^\dagger}{\sqrt{2}}\\
    \text{エ:} &\qquad \hat{p} = -i\sqrt{m\hbar\omega}\frac{\hat{a}-\hat{a}^\dagger}{\sqrt{2}}\\
    \text{オ:} &\qquad \mathcal{H} = \hbar\omega\qty(\hat{a}^\dagger \hat{a}+\frac{1}{2})\\
    \text{カ:} &\qquad E_0 = \frac{1}{2}\hbar\omega\\
    \text{キ:} &\qquad \ket{\phi_{n+1}}=\frac{1}{\sqrt{n+1}}\hat{a}^\dagger\ket{\phi_n}\\
    \text{ク:} &\qquad E_n = \hbar\omega\qty(n+\frac{1}{2})
\end{align}

\subsection{}
シュレディンガー方程式より
\begin{align*}
    i\hbar\dv{t}\ket{\psi(t)} &= \mathcal{H}\\
    \sum_{n=0}^{\infty}i\hbar\dv{t}c_n(t)\ket{\phi_n}
    &=\sum_{n=0}^{\infty}\hbar\omega\qty(\hat{a}^\dagger\hat{a}\frac{1}{2})c_n(t)\ket{\phi_n}\\
    \sum_{n=0}^{\infty}\qty[i\hbar\dv{c_n(t)}{t}-\hbar\omega\qty(n+\frac{1}{2})]\ket{\phi_n}&=0
\end{align*}
これより\(c_n(t)\)の時間発展の微分方程式が得られ、これを解くと
\begin{align*}
    c_n(t) = c_n(0) \exp[-i\qty(n+\frac{1}{2})\omega t]
\end{align*}
となる。

\subsection{}
位置の期待値は
\begin{equation}
    \begin{aligned}[b]
        \bra{\phi_n}\hat{x}\ket{\phi_n}
        &=\sqrt{\frac{\hbar}{2m\omega}}\bra{\phi_n}\qty(\hat{a}+\hat{a}^\dagger)\ket{\phi_n}\\
        &=\sqrt{\frac{\hbar}{2m\omega}}\sqrt{n+1}\Bigl[\bra{\phi_{n+1}}\ket{\phi_n}+\bra{\phi_{n}}\ket{\phi_{n+1}}\Bigr]\\
        &=0
    \end{aligned}
\end{equation}
位置の2乗の期待値は
\begin{equation}
    \begin{aligned}[b]
        \bra{\phi_n}\hat{x}^2\ket{\phi_n}
        &=\frac{\hbar}{2m\omega}\bra{\phi_n}\qty(
            \hat{a}^\dagger\hat{a}^\dagger+\hat{a}^\dagger\hat{a}+\hat{a}\hat{a}^\dagger+\hat{a}\hat{a})
            \ket{\phi_n}\\
        &= \frac{\hbar}{2m\omega}\bra{\phi_n}\qty(2\hat{a}^\dagger + 1)\ket{\phi_n}\\
        &= \frac{\hbar}{m\omega}\qty(n+\frac{1}{2})
    \end{aligned}
\end{equation}
となる。
また同様にして運動量の方を計算していく。
運動量の期待値は
\begin{equation}\begin{aligned}[b]
    \bra{\phi_n}\hat{p}\ket{\phi_n}
    &= -i\sqrt{\frac{m\hbar\omega}{2}}\bra{\phi_n}\qty(\hat{a}-\hat{a}^\dagger)\ket{\phi_n}
    &= -i\sqrt{\frac{m\hbar\omega}{2}}\sqrt{n+1}\Bigl[\bra{\phi_{n+1}}\ket{\phi_n}-\bra{\phi_{n}}\ket{\phi_{n+1}}\Bigr]\\
    &= 0
\end{aligned}\end{equation}

運動量の2乗の期待値は
\begin{equation}\begin{aligned}[b]
    \bra{\phi_n}\hat{p}^2\ket{\phi_n}
    &=-\frac{m\hbar\omega}{2}\bra{\phi_n}\qty(
        \hat{a}^\dagger\hat{a}^\dagger-\hat{a}^\dagger\hat{a}-\hat{a}\hat{a}^\dagger+\hat{a}\hat{a})
        \ket{\phi_n}\\
    &= \frac{m\hbar\omega}{2}\bra{\phi_n}\qty(2\hat{a}^\dagger + 1)\ket{\phi_n}\\
    &= \frac{m\hbar\omega}{2}\qty(n+\frac{1}{2})
\end{aligned}\end{equation}
よって
\begin{equation}\begin{aligned}[b]
    (\Delta x\cdot \Delta p)^2 &= \frac{\hbar}{m\omega}\qty(n+\frac{1}{2})\times\frac{m\hbar\omega}{2}\qty(n+\frac{1}{2})\\
    \Delta x\cdot \Delta p &= \hbar\qty(n+\frac{1}{2})
\end{aligned}\end{equation}

\subsection{}
\begin{equation}\begin{aligned}[b]
    \hat{a}\ket{\alpha}
    &= e^{-\abs{\alpha}^2/2}\sum_{n=0}^\infty\frac{\alpha^n}{\sqrt{n!}}\hat{a}\ket{\phi_n}\\
    &= e^{-\abs{\alpha}^2/2}\sum_{n=1}^\infty\frac{\alpha^n}{\sqrt{n!}}\sqrt{n}\ket{\phi_{n-1}}\\
    &= \alpha e^{-\abs{\alpha}^2/2}\sum_{n=1}^\infty\frac{\alpha^{n-1}}{\sqrt{(n-1)!}}\ket{\phi_{n-1}}\\
    &= \alpha\ket{\alpha}
\end{aligned}\end{equation}
より\(\ket{\alpha}\)は\(\hat{a}\)の固有状態。

このことを使うとこの状態での位置の期待値は
\begin{equation}\begin{aligned}[b]
    \bra{\alpha}\hat{x}\ket{\alpha}
    &= \sqrt{\frac{\hbar}{2m\omega}}\bra{\alpha}\qty(\hat{a}+\hat{a}^\dagger)\ket{\alpha}\\
    &=\sqrt{\frac{\hbar}{2m\omega}}\qty(\alpha+\alpha^*)\\
    &= \sqrt{\frac{2\hbar}{m\omega}}r\cos\theta
\end{aligned}\end{equation}
運動量の期待値は
\begin{equation}\begin{aligned}[b]
    \bra{\alpha}\hat{p}\ket{\alpha}
    &= -i\sqrt{\frac{m\hbar\omega}{2}}\bra{\alpha}\qty(\hat{a}-\hat{a}^\dagger)\ket{\alpha}\\
    &= -i\sqrt{\frac{m\hbar\omega}{2}}\qty(\alpha-\alpha^*)\\
    &= \sqrt{2m\hbar\omega}r\sin\theta
\end{aligned}\end{equation}

\subsection{}
設問2より\(\ket{\phi_n}\)の時間発展は
\begin{equation}\begin{aligned}[b]
    \ket{\phi_n(t)} = c_n(0)\exp[-i\qty(n+\frac{1}{2})\omega t]
\end{aligned}\end{equation}
と書けるので\(\ket{\psi(0)}=\ket{\alpha=r_0}\)の時間発展は
\begin{equation}\begin{aligned}[b]
    \ket{\psi(t)}
    &= e^{-r_0^2/2}\sum_{n=0}^{\infty}\frac{r_0^n}{\sqrt{n!}}e^{-in\omega t}e^{-i\omega t/2}\ket{\phi_n}\\
    &= e^{-i\omega t/2}e^{-r_0^2/2}\sum_{n=0}^{\infty}\frac{\qty(r_0e^{-i\omega t})^n}{\sqrt{n!}}\ket{\phi_n}\\
    &= e^{-i\omega t/2}\ket{\alpha=r_0e^{-i\omega t}}
\end{aligned}\end{equation}
となる。
先頭の\(e^{-i\omega t/2}\)の位相は物理量の期待値には影響しないので無視できる。
すると問題の無次元化した位置の期待値は設問4で求めたものに\(r=r_0,\,\theta = -\omega t\)を入れれば良いことがわかるので、
\begin{align}
    \bra{\psi(t)}\hat{X}\ket{\psi(t)} &= \sqrt{2}r_0\cos\omega t\\
    \bra{\psi(t)}\hat{P}\ket{\psi(t)} &= -\sqrt{2}r_0\sin\omega t
\end{align}
となる。
位置と運動量の期待値を位相空間でみると、時計回りに等速円運動するという
古典系と同じ運動になっていることがわかる。

\subsection*{感想}
コヒーレント状態の話。
場の量子論とか量子光学でやる計算で、
交換子と固有状態をうまく組み合わせると楽にできるというやつ。

\subsection{おまけ}
コヒーレント状態の復習をしたくなったので、
この問題の続きとなるものをやってみる。
\newline
\begin{tcolorbox}[colbacktitle=white, coltitle=black, colback=white, title=おまけ1]
    コヒーレント状態の位置の分散\(\Delta X^2\)と運動量の分散\(\Delta P^2\)を求めよ。
    そしてこれが極小不確定状態である、つまり\(\Delta X \Delta P = 1/2\)となることを確かめよ。
\end{tcolorbox}
まず位置の2乗平均については、
\begin{equation}\begin{aligned}[b]
    \bra{\alpha}\hat{X}^2\ket{\alpha}
    &= \frac{1}{2}\bra{\alpha}\qty(
        \hat{a}^\dagger\hat{a}^\dagger+\hat{a}^\dagger\hat{a}+\hat{a}\hat{a}^\dagger+\hat{a}\hat{a})\ket{\alpha}\\
    &= \frac{1}{2}\bra{\alpha}\qty(
        \hat{a}^\dagger\hat{a}^\dagger+2\hat{a}^\dagger\hat{a}+\hat{a}\hat{a}+1)\ket{\alpha}\\
    &= \frac{1}{2}\qty(\alpha^2+\alpha^{*2}+2\abs{\alpha}^2)+\frac{1}{2}\\
    &= r^2(1+\cos2\theta)+\frac{1}{2} = 2r^2\cos^2\theta+\frac{1}{2}.
\end{aligned}\end{equation}
これより位置の分散は
\begin{align*}
    \Delta X^2 = \ev{X^2}-\ev{X}^2 = \frac{1}{2}
\end{align*}
運動量の2乗平均については
\begin{equation}\begin{aligned}[b]
    \bra{\alpha}\hat{P}^2\ket{\alpha}
    &= -\frac{1}{2}\bra{\alpha}\qty(
        \hat{a}^\dagger\hat{a}^\dagger-\hat{a}^\dagger\hat{a}-\hat{a}\hat{a}^\dagger+\hat{a}\hat{a})\ket{\alpha}\\
    &= -\frac{1}{2}\qty(\alpha^2+\alpha^{*2}-2\abs{\alpha}^2-1)\\
    &= r^2(1-\cos2\theta) +\frac{1}{2}= 2r^2\sin^2\theta + \frac{1}{2}.
\end{aligned}\end{equation}
これより運動量の分散は
\begin{equation}\begin{aligned}[b]
    \Delta P^2 = \ev{P^2}-\ev{P}^2 = \frac{1}{2}
\end{aligned}\end{equation}
よって
\begin{equation}\begin{aligned}[b]
    \Delta X \Delta P = \frac{1}{2}
\end{aligned}\end{equation}
これより、位置と運動量の期待値自体は古典系のときと同じであるが、
状態は位相空間の1点ではなくて、直径\(1/2\)程度のぼんやりと広がった状態であることがわかる。
\newline
\begin{tcolorbox}[colbacktitle=white, coltitle=black, colback=white, title=おまけ2]
    レーザー発振のハミルトニアンは
    \begin{equation}\begin{aligned}[b]
        \mathcal{H}_{\text{laser}} \propto i(\alpha\hat{a}^\dagger-\alpha^* \hat{a})
    \end{aligned}\end{equation}
    というように書けるらしい。
    第一項が誘導放出、第二項が誘導吸収のイメージである。
    これによるユニタリ発展つまり
    \begin{equation}\begin{aligned}[b]
        \hat{D}(\alpha) = e^{-i\mathcal{H}_{\text{laser}}}
    \end{aligned}\end{equation}
    が真空状態からコヒーレント状態を生成する、つまり
    \begin{equation}\begin{aligned}[b]
        \ket{\alpha} = \hat{D}(\alpha)\ket{0} =e^{\alpha\hat{a}^\dagger-\alpha^* \hat{a}}\ket{0}
    \end{aligned}\end{equation}
    となることを確かめよ。この\(\hat{D}(\alpha)\)は変位演算子という。
    その際に\([A,[A,B]]=[B,[A,B]]=0\)の際のBCH公式
    \begin{equation}\begin{aligned}[b]
        e^{A+B}=e^{A}e^{B}e^{-[A,B]/2}
    \end{aligned}\end{equation}
    を使うとよい。
\end{tcolorbox}
BCH公式を使って計算を進めていくと
\begin{equation}\begin{aligned}[b]
    e^{\alpha\hat{a}^\dagger-\alpha^* \hat{a}}\ket{0}
    &= e^{\alpha\hat{a}^\dagger} e^{-\alpha^*\hat{a}}e^{-[\alpha \hat{a}^\dagger,-\alpha^*\hat{a}]/2}\ket{0}\\
    &= e^{-\abs{\alpha}^2/2}e^{\alpha\hat{a}^\dagger} e^{-\alpha^*\hat{a}}\ket{0}\\
    &= e^{-\abs{\alpha}^2/2}e^{\alpha\hat{a}^\dagger}\ket{0}\\
    &= e^{-\abs{\alpha}^2/2}\sum_{n=0}^{\infty}\frac{\alpha^n}{n!}\hat{a}^{\dagger n}\ket{0}\\
    &= e^{-\abs{\alpha}^2/2}\sum_{n=0}^{\infty}\frac{\alpha^n}{n!}\sqrt{n!}\ket{n}\\
    &= e^{-\abs{\alpha}^2/2}\sum_{n=0}^{\infty}\frac{\alpha^n}{\sqrt{n!}}\ket{n}\\
    &= \ket{\alpha}
\end{aligned}\end{equation}

\begin{tcolorbox}[colbacktitle=white, coltitle=black, colback=white, title=おまけ3]
    スクイーズ状態についてなんか
\end{tcolorbox}


\clearpage
\section{統計力学: BEC}
\subsection{}
状態\(\epsilon_i\)の状態が\(n_i\)個ある確率は
\begin{equation}\begin{aligned}[b]
    \frac{e^{-\beta(\epsilon_i-\mu)n_i}}{\Theta}
\end{aligned}\end{equation}
であるので分布関数は
\begin{equation}\begin{aligned}[b]
    f(\epsilon_i) =\sum_{n_i}^{\infty}n_i \frac{e^{-\beta(\epsilon_i-\mu)n_i}}{\Theta}
\end{aligned}\end{equation}


\end{document}