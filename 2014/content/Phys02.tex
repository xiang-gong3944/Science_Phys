\documentclass[../../sp_2014.tex]{subfiles}

\graphicspath{{./image/}}

\begin{document}
\section{統計力学: 1次元ゴム}


\subsection*{感想}
前半の特徴はスターリングの公式ではなく、鞍点法で状態数を求めることのように感じる。

設問5で分配関数を状態数のフーリエ変換であることに気付くのがなかなか難しい。
田崎統計とかに分配関数は状態数のラプラス変換であるということを知ってれば、
ラプラス変換の\(s\)を解析接続のようなことをするとフーリエ変換と読めるというイメージでもありなのかなぁ。
もしくは虚時間が温度に対応する的なことからも連想できるのかなぁ。

\end{document}