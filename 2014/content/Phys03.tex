\documentclass[../../sp_2014.tex]{subfiles}

\graphicspath{{./image/}}

\begin{document}
\section{電磁気学: 相対論的電磁気学}
\subsection{}
\(\abs{x}\gg a\)では並べられた荷電粒子群は線電荷密度\(q/a\)を持った直線とみなせる。
このとき、直線を中心とした\(r=x,\,0\geq z\geq l\)の円筒のについて
ガウスの法則を使うと、対称性より円筒の側面だけ考えればよく、
\begin{equation}\begin{aligned}[b]
     2\pi xlE_x &= \frac{ql}{a\epsilon_0}\\
     E_x &= \frac{q}{2\pi\epsilon_0ax}
\end{aligned}\end{equation}
となる。電場に他の成分はない。
また、磁場に関しては今の系に電流はないので
\begin{equation}\begin{aligned}[b]
    \vb*{B}=0
\end{aligned}\end{equation}

\subsection{}
2つの隣接した荷電粒子を考える。
ぞれぞれの位置を\(\mathcal{O}\)系で\(z=z_0,z_0+a\)にあるあるものとする。
これらをLorentz変換すると
\begin{equation}\begin{aligned}[b]
    z_0 \to \gamma(z_0-v_zt), \quad z_0+a \to \gamma(z_0+a-v_zt)
\end{aligned}\end{equation}
となる。
これらの差が\(\mathcal{O}'\)系での荷電粒子の間隔\(a'\)なので
\begin{equation}\begin{aligned}[b]
    a' = \gamma(z_0+a-v_zt)-\gamma(z_0-v_zt) = \gamma a
\end{aligned}\end{equation}

\subsection{}
電場は設問1の電荷密度\(q/a\)が\(\gamma q/a\)に変わったものとなるので
\begin{equation}\begin{aligned}[b]
    E_x' =\gamma \frac{qx'}{2\pi\epsilon_0a(x'^2+y'^2)},\quad
    E_y' =\gamma \frac{qy'}{2\pi\epsilon_0a(x'^2+y'^2)},\quad
    E_z' = 0
\end{aligned}\end{equation}
磁場について、\(\mathcal{O}'\)系では\(-z\)方向に速度\(v\)で電荷密度\(\gamma q/a\)が動いているので
\(-\gamma qv_z/a\)の電流が流れているとみなせる。
アンペールの法則より磁場は円筒座標の\(\varphi\)成分だけ残って
\begin{equation}\begin{aligned}[b]
    2\pi\sqrt{x'^2+y'^2}B_\varphi' &= -\gamma\frac{\mu_0qv_z}{a}\\
    B_\varphi' &= -\frac{\beta\gamma}{c}\frac{q}{2\pi\varepsilon_0 a\sqrt{x'^2+y'^2}}
\end{aligned}\end{equation}
よって直交座標で表すと
\begin{equation}\begin{aligned}[b]
    B_x' = \frac{\beta\gamma}{c}\frac{q y'}{2\pi\varepsilon_0 a(x'^2+y'^2)},\quad
    B_y' = -\frac{\beta\gamma}{c}\frac{q x'}{2\pi\varepsilon_0 a(x'^2+y'^2)},\quad
    B_z' = 0
\end{aligned}\end{equation}

\subsection{}
電磁場のLorentz変換の式より
\(E_x\)は
\begin{equation}\begin{aligned}[b]
    E_x &= \gamma(E_x'+c\beta B_y')
    = \gamma\qty(\gamma \frac{qx'}{2\pi\epsilon_0a(x'^2+y'^2)}-\beta^2\gamma\frac{qx'}{2\pi\varepsilon_0 a(x'^2+y'^2)})\\
    &= \gamma^2(1-\beta^2) \frac{qx}{2\pi\epsilon_0a(x^2+y^2)}
    = \frac{qx}{2\pi\epsilon_0a(x^2+y^2)}
\end{aligned}\end{equation}
\(E_y\)は
\begin{equation}\begin{aligned}[b]
    E_y &= \gamma(E_y'-c\beta B_x')
    = \gamma\qty(\gamma \frac{qy'}{2\pi\epsilon_0a(x'^2+y'^2)}-\beta^2\gamma\frac{qy'}{2\pi\varepsilon_0 a(x'^2+y'^2)})\\
    &= \gamma^2(1-\beta^2) \frac{qy}{2\pi\epsilon_0a(x^2+y^2)}
    = \frac{qy}{2\pi\epsilon_0a(x^2+y^2)}
\end{aligned}\end{equation}
\(E_z\)は
\begin{equation}\begin{aligned}[b]
    E_z = E_z' =0
\end{aligned}\end{equation}
\(B_x\)は
\begin{equation}\begin{aligned}[b]
    B_x &= \gamma\qty(B_x'-\frac{\beta}{c}E_y')
    = \gamma\qty( \frac{\beta\gamma}{c}\frac{q y'}{2\pi\varepsilon_0 a(x'^2+y'^2)}- \frac{\beta\gamma}{c}\frac{q y'}{2\pi\varepsilon_0 a(x'^2+y'^2)})
    = 0
\end{aligned}\end{equation}
\(B_y\)は
\begin{equation}\begin{aligned}[b]
    B_x &= \gamma\qty(B_y'+\frac{\beta}{c}E_x')
    = \gamma\qty(-\frac{\beta\gamma}{c}\frac{q x'}{2\pi\varepsilon_0 a(x'^2+y'^2)}+ \frac{\beta\gamma}{c}\frac{q x'}{2\pi\varepsilon_0 a(x'^2+y'^2)})
    = 0
\end{aligned}\end{equation}
\(B_z\)は
\begin{equation}\begin{aligned}[b]
    B_z = B_z' =0
\end{aligned}\end{equation}
となりコンシステントな結果が得られる。

\subsection{}
\(\mathcal{O}'\)において、電流はなく、中心に点電荷があるだけであるので
\begin{equation}\begin{aligned}[b]
    E_x' = \frac{qx'}{4\pi\epsilon_0a(x'^2+y'^2+z'^2)^{3/2}},\quad
    E_y' = \frac{qy'}{4\pi\epsilon_0a(x'^2+y'^2+z'^2)^{3/2}},\quad
    E_z' = \frac{qz'}{4\pi\epsilon_0a(x'^2+y'^2+z'^2)^{3/2}}\\
\end{aligned}\end{equation}
\begin{equation}\begin{aligned}[b]
    \vb*{B} = 0
\end{aligned}\end{equation}
\subsection{}
\(x'=x,y'=y=0,z'=\gamma(z-\beta ct)=-\gamma v_zt\)に注意して電磁場のLorentz変換をすると
\begin{equation}\begin{aligned}[b]
    E_x = \gamma\frac{qx}{4\pi\epsilon_0(x^2+\gamma^2v_z^2 t^2)^{3/2}},\quad
    E_y = 0,\quad
    E_z = \gamma\frac{-q v_z t}{4\pi\epsilon_0(x^2+\gamma^2v_z^2 t^2)^{3/2}}
\end{aligned}\end{equation}
\begin{equation}\begin{aligned}[b]
    B_x = 0,\quad
    B_y = \gamma\frac{\mu_0 q x}{4\pi(x^2+\gamma^2v_z^2t^2)^{3/2}},\quad
    B_z = 0
\end{aligned}\end{equation}


\subsection*{感想}
荷電粒子が直線状に並んでいるという系から、
電磁場のLorentz変換が出るという話。
久しぶりにやったもんだから、思い出すのに時間がかかった。
問題文にLorentz変換の式が出てるおかげで、
答えがあってるかのチェックはやりやすかった。

\end{document}