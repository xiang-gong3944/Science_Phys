\documentclass[../../sp_2014.tex]{subfiles}

\graphicspath{{./image/}}

\begin{document}
\section{量子力学: 周期境界条件・デルタ関数ポテンシャル・摂動論}
\subsection{}
シュレーディンガー方程式は
\begin{equation}\begin{aligned}[b]
    -\frac{\hbar^2}{2m}\dv[2]{x}\psi(x)=E\psi(x)
\end{aligned}\end{equation}
\(E<0\)のときの解は
\begin{equation}\begin{aligned}[b]
    \psi(x) = Ae^{\kappa x}+Be^{-\kappa x}, \qquad \kappa = \frac{\sqrt{-2mE}}{\hbar}
\end{aligned}\end{equation}
となる。
周期的境界条件より
\begin{equation}\begin{aligned}[b]
    0&=\psi(x+2\pi L)-\psi(x)\\
    &= Ae^{2\pi\kappa L}e^{\kappa x} + Be^{-2\pi\kappa L}e^{-\kappa x}-Ae^{\kappa x}+Be^{-\kappa x}\\
    &= A(e^{2\pi\kappa L}-1)e^{\kappa x} + B(e^{-2\pi\kappa L}-1)e^{-\kappa x}
\end{aligned}\end{equation}
これを満たす\(\kappa\)は\(\kappa=0\)のみであるので、\(E<0\)としたのと反するのでこのような解はない。

\(E>0\)のときの解は
\begin{equation}\begin{aligned}[b]
    \psi(x) = Ae^{ik x}+Be^{-ik x}, \qquad k = \frac{\sqrt{2mE}}{\hbar}
\end{aligned}\end{equation}
となる。
周期的境界条件より
\begin{equation}\begin{aligned}[b]
    0&=\psi(x+2\pi L)-\psi(x)\\
    &= Ae^{2\pi ikL}e^{ik x} + Be^{-2\pi ikL}e^{-ikx}-Ae^{ikx}+Be^{-ikx}\\
    &= A(e^{2\pi ikL}-1)e^{ikx} + B(e^{-2\pi ikL}-1)e^{-\kappa x}
\end{aligned}\end{equation}
これより
\begin{equation}\begin{aligned}[b]
    e^{2\pi ikL} &= 1\\
    k &= \frac{n}{L},\qquad(n\text{は0以上の整数})\\
    \rightarrow& \quad  E= \frac{\hbar^2k^2}{2m}=\frac{\hbar^2 n^2}{2mL^2}
\end{aligned}\end{equation}
波動関数の係数\(A,B\)が決まらないのでこれは縮退している。

\subsection{}
シュレーディンガー方程式の両辺を微小区間\((-\epsilon,\epsilon)\)で積分して\(\epsilon\to0+\)とすると
\begin{equation}\begin{aligned}[b]
    \int_{-\epsilon}^\epsilon dx\qty[-\frac{\hbar^2}{2m}\dv[2]{x}+\frac{\hbar^2 v}{2m}\delta(x)]\psi(x)
    &= \int_{-\epsilon}^{\epsilon} dx\, E\psi(x)\\
    \frac{\hbar^2}{2m}\lim_{\epsilon\to0+}-\qty(\dv{\psi(\epsilon)}{x}-\dv{\psi(-\epsilon)}{x})+\frac{\hbar^2v}{2m}\psi(0)&=0\\
    \lim_{\epsilon\to0}\qty(\psi'(\epsilon)-\psi'(-\epsilon))&=v\psi(0)
\end{aligned}\end{equation}

\subsection{}
\(x\in (0,2\pi L)\)でのシュレディンガー方程式に与えらえた波動関数を入れると
\begin{equation}\begin{aligned}[b]
    -\frac{\hbar^2}{2m}\dv[2]{x}\qty(e^{\kappa x}+Ae^{-\kappa x}) &= -\frac{\hbar^2 \kappa^2}{2m}\qty(e^{\kappa x}+Ae^{-\kappa x})
    &= -\frac{\hbar^2\kappa^2}{2m}\psi(x)
\end{aligned}\end{equation}
より確かに\(E=-\hbar^2\kappa^2/2m\)の解になっているのがわかる。
次に\(x=0\)での境界条件を考える。
周期的境界条件より\(x=0\)と\(x=2\pi L\)での波動関数の値は同じであるので
\begin{equation}\begin{aligned}[b]
    \psi(0) &= \psi(2\pi L)\\
    1+A &= e^{2\pi\kappa L}+Ae^{-2\pi\kappa L}\\
    \qty(e^{2\pi\kappa L}-1)\qty(e^{2\pi\kappa L}-A) &= 0
\end{aligned}\end{equation}
\(e^{2\pi\kappa L}=1\)となるのは\(\kappa=0\)であるので不適。
よって\(A=e^{2\pi\kappa L}\)
また、設問2で求めた条件を周期的境界条件と合わせると
\begin{equation}\begin{aligned}[b]
    \psi'(0)-\psi'(2\pi L) &= v\psi(0)\\
    \kappa(1-A)-\kappa(e^{2\pi\kappa L}-Ae^{-2\pi\kappa L}) &= (1+A)v\\
    v &= \frac{2(1-e^{2\pi\kappa L})}{1+e^{2\pi\kappa L}}\kappa \\
    &= -2\kappa\tanh(\pi\kappa L)
\end{aligned}\end{equation}
となる。
\(\kappa\tanh(\pi\kappa L)\)は単調減少で\(\kappa=0\)のとき\(v=0\)なので\(v\)と\(\kappa\)は1対1に対応する

\subsection{}
波動関数は設問1で求めた形になるので、
波動関数の規格化定数はおいておいて、
\begin{equation}\begin{aligned}[b]
    \psi(x)=e^{ikx}+Ae^{-ikx}
\end{aligned}\end{equation}
とする。
\(x=0\)と\(x=2\pi L\)で波動関数の値が連続であるであることより
\begin{equation}\begin{aligned}[b]
    \psi(0) &= \psi(2\pi L)\\
    1+A &= e^{2\pi ikL}+Ae^{-2\pi ikL}\\
    \qty(e^{2\pi ikL}-1)\qty(e^{2\pi ikL}-A) &= 0
\end{aligned}\end{equation}

まず\(e^{2\pi ikL}=1\)つまり\(\kappa =n/L\)のように設問1と同じ条件のとき、
微分係数の条件から
\begin{equation}\begin{aligned}[b]
    \psi'(0)-\psi'(2\pi L) &= v\psi(0)\\
    ik(1-A)-ik(e^{2\pi ikL}-Ae^{-2\pi ikL})&=v(1+A)
    A &= -1
\end{aligned}\end{equation}
よってこのときのシュレーディンガー方程式の解は
\begin{equation}\begin{aligned}[b]
    \psi(x) \propto \sin(\frac{\pi x}{L}), \quad E=\frac{\hbar^2n^2}{2mL^2}
\end{aligned}\end{equation}
となり、ポテンシャルの影響を受けない解となる。

次に\(e^{2\pi ikL}=A\)のとき、
微分係数に関する境界条件より
\begin{equation}\begin{aligned}[b]
    \psi'(0)-\psi'(2\pi L) &= v\psi(0)\\
    ik(1-A)-ik(e^{2\pi ikL}-Ae^{-2\pi ikL})&=v(1+A)\\
    v &= \frac{2ik\qty(1-e^{2\pi ikL})}{1+e^{2\pi ikL}}\\
    &= 2k\tan(\pi kL)
\end{aligned}\end{equation}
これもグラフを書くことにより\(v\)をあたえると\(k\)が量子化しているのがわかる。
k について解くことはできないので波動関数の形だけ示しておくと
\begin{equation}\begin{aligned}[b]
    \psi(x)\propto e^{ikx}+e^{2\pi ikL}e^{-ikx}
\end{aligned}\end{equation}
となる。

\subsection{}
設問4で得られた解の内、ポテンシャルの影響がないものはエネルギー固有値の補正も受けない。
影響を受けた方について、
摂動論の一般論より、\(v\)の1次摂動の補正エネルギー\(\Delta E\)は
\begin{equation}\begin{aligned}[b]
    \Delta E &= \frac{\int dx \psi^*(x)\frac{\hbar^2v}{2m}\delta(x)\psi(x)}{\int dx \psi^*(x)\psi(x)}\\
    &= \frac{\hbar^2 v}{2m}\frac{\int dx \,4\cos^2(kx-\pi kL)\delta(x)}{\int dx \,4\cos^2(kx-\pi kL)}
    &= \frac{\hbar^2\cos^2(\pi kL)}{\pi mL}v
\end{aligned}\end{equation}
となる。

\subsection*{感想}
ただ単に周期境界条件・デルタ関数ポテンシャル・摂動論を組み合わせてみましたという感じ。
時間に余裕はないが、バンドギャップの話につながりそう。


\end{document}