\documentclass[../../sp_2014.tex]{subfiles}

\graphicspath{{./image/}}

\begin{document}
\section{原子核実験: コンプトン散乱・硬X線}
\subsection{}
光子の波長とエネルギーの関係は\(\lambda = 2\pi\hbar c/E\)より
\begin{equation}\begin{aligned}[b]
    \lambda = 2\pi \frac{200\,\si{MeV fm}}{0.5\,\si{MeV}}= 4.0\,\si{nm}
\end{aligned}\end{equation}

\subsection{}
エネルギー保存則は
\begin{equation}\begin{aligned}[b]
    \frac{hc}{\lambda}+m_ec^2 = \frac{hc}{\lambda'}+\sqrt{m_e^2c^4+c^2p_e^2}
\end{aligned}\end{equation}
運動量保存則は
\begin{equation}\begin{aligned}[b]
    \frac{h}{\lambda} &= \frac{h}{\lambda'}\cos\theta + p_e\cos\psi\\
    0&= \frac{h}{\lambda}\sin\theta + p_e \sin\psi
\end{aligned}\end{equation}

\subsection{}
運動量保存則を君合わせ\(\psi\)を消去すると
\begin{equation}\begin{aligned}[b]
    p_e^2&=h^2\qty(\frac{1}{\lambda}-\frac{\cos\theta}{\lambda'})^2 + \frac{h^2}{\lambda'^2}\\
    \frac{p_e^2}{h^2}&= \frac{1}{\lambda^2}+\frac{h^2}{\lambda'^2}-\frac{2}{\lambda\lambda'}\cos\theta
\end{aligned}\end{equation}
これをエネルギー保存則の式に入れていくと
\begin{equation}\begin{aligned}[b]
    \frac{hc}{\lambda}-\frac{hc}{\lambda'}+m_ec^2 &= \sqrt{m_e^2c^4+c^2p_e^2}\\
    \qty(\frac{1}{\lambda}-\frac{1}{\lambda})^2+\frac{2m_ec}{h}\qty(\frac{1}{\lambda}-\frac{1}{\lambda'})&= \frac{p_e^2}{h^2}
    \frac{m_e c}{h}\frac{\lambda'-\lambda}{\lambda\lambda'}&=-\frac{\cos\theta}{\lambda\lambda'}\\
    \lambda'-\lambda &= \frac{h}{m_ec}(1-\cos\theta)
\end{aligned}\end{equation}


\subsection*{感想}
X線自体は物性実験・素核実験で必ず使うものなので、
実験の人は抑えておいたほうがよい内容なのかと思ったけど、
後半が全く分からない。


\end{document}