\documentclass[../../sp_2015.tex]{subfiles}

\graphicspath{{./image/}}

\begin{document}
\section{統計力学: 固体の比熱}
\subsection{}
この系の古典的な分配関数は
\begin{equation}\begin{aligned}[b]
    Z &= \frac{1}{(2\pi\hbar)^N}\qty(\int_{-\infty}^{\infty}dp\int_{-\infty}^{\infty}dx\,e^{-\beta p^2/2m}e^{-\beta m\omega(x-a)^2/2})^N\\
    &= \frac{1}{(2\pi\hbar)^N}\qty(\sqrt{\frac{2m\pi}{\beta}}\sqrt{\frac{2\pi}{\beta m\omega^2}})^N
    =\qty(\frac{k_BT}{\hbar\omega})^N
\end{aligned}\end{equation}
よって内部エネルギーは
\begin{equation}\begin{aligned}[b]
    \frac{U}{N}=-\frac{1}{N}\pdv{\beta}\ln Z = k_BT^2\pdv{T}\ln\frac{k_BT}{\hbar\omega} = kB_T
\end{aligned}\end{equation}
比熱は
\begin{equation}\begin{aligned}[b]
    C = \dv{U}{T}=k_B
\end{aligned}\end{equation}
となる。
位置の期待値を求める。
1つの粒子の分配関数を考えることにより\(\ev{x_1}\)は
\begin{equation}\begin{aligned}[b]
    \ev{x_i} &= \frac{\int dp\,e^{-\beta p^2/2m}\int dx\, xe^{-\beta m\omega^2(x-a)^2/2}}{\int dp\,e^{-\beta p^2/2m}\int dx\, e^{-\beta m\omega^2(x-a)^2/2}}\\
    &= a
\end{aligned}\end{equation}
そして\(\ev{x_1^2}\)は
\begin{equation}\begin{aligned}[b]
    \ev{x_i} &= \frac{\int dp\,e^{-\beta p^2/2m}\int dx\, x^2e^{-\beta m\omega^2(x-a)^2/2}}{\int dp\,e^{-\beta p^2/2m}\int dx\, e^{-\beta m\omega^2(x-a)^2/2}}\\
    &= \frac{\int dx\, \qty[(x-a)^2+2a(x-a)+a^2]e^{-\beta m\omega^2(x-a)^2/2}}{\int dx\, e^{-\beta m\omega^2(x-a)^2/2}}\\
    &= \frac{k_BT}{m\omega^2}+a^2
\end{aligned}\end{equation}
となる。よって分散は
\begin{equation}\begin{aligned}[b]
    \ev{x_1^2}-\ev{x_1}^2=\frac{k_BT}{m\omega^2}
\end{aligned}\end{equation}

\subsection{}
1つの質点の分散関係は
\begin{equation}\begin{aligned}[b]
    Z = \sum_{n=0}^{\infty}\bra{n}e^{-\beta H}\ket{n}
    = \sum_{n=0}^{\infty}e^{-\beta\hbar\omega(n+1/2)}
    =\frac{e^{-\beta\hbar\omega/2}}{1-e^{-\beta\hbar\omega}}
    = \frac{1}{2\sinh(\beta\hbar\omega/2)}
\end{aligned}\end{equation}
これより位置の期待値は
\begin{equation}\begin{aligned}[b]
    \ev{x} = \frac{1}{Z}\sum_{n=0}^{\infty}\bra{n}Xe^{-\beta H}\ket{n}
    = \frac{1}{Z}\sum_{n=0}^{\infty}e^{-\beta\hbar\omega(n+1/2)}
        \sqrt{\frac{\hbar}{2m\omega}}\bra{n}\qty(\sqrt{n}\ket{n-1}+\sqrt{n+1}\ket{n+1})
    = 0
\end{aligned}\end{equation}
なので分散は
\begin{equation}\begin{aligned}[b]
    \ev{x^2} &= \frac{1}{Z}\sum_{n=0}^{\infty}\bra{n}Xe^{-\beta H}\ket{n}\\
    &= \frac{1}{Z}\sum_{n=0}^{\infty}e^{-\beta\hbar\omega(n+1/2)}
        \frac{\hbar}{2m\omega}\bra{n}\qty(\sqrt{n(n-1)}\ket{n-2}+n\ket{n}+(n+1)\ket{n}+\sqrt{(n+1)(n+2)}\ket{n+2})\\
    &= \frac{1}{Z}\sum_{n=0}^{\infty} \frac{1}{m\omega^2}\hbar\omega\qty(n+\frac{1}{2})e^{-\beta\hbar(n+1/2)}
    = -\frac{1}{m\omega^2}\pdv{\beta}\ln Z =\frac{\hbar}{2m\omega}\coth\frac{\beta\hbar\omega}{2}
\end{aligned}\end{equation}

また、内部エネルギーは
\begin{equation}\begin{aligned}[b]
    U = -\pdv{\beta}\ln Z
    = \pdv{\beta} \ln\sinh\frac{\beta\hbar\omega}{2}
    =\frac{\hbar\omega}{2}\coth\frac{\beta\hbar\omega}{2}
\end{aligned}\end{equation}
比熱は
\begin{equation}\begin{aligned}[b]
    C = \dv{U}{T}
    = -k_B\beta^2\dv{\beta}\coth\frac{\beta\hbar\omega}{2}
    = k_B\qty(\frac{\hbar\omega}{2k_BT})^2\qty[\frac{\cosh^2\beta\hbar\omega/2}{\sinh^2\beta\hbar\omega/2}-1]
    = k_B \qty(\frac{\hbar\omega/2k_BT}{\sinh\hbar\omega/2k_BT})^2
\end{aligned}\end{equation}
となる。
\(T\to\infty\)の高温極限では\(\sinh\hbar\omega/2k_BT\simeq\hbar\omega/2k_BT\)より
\begin{equation}\begin{aligned}[b]
    C = k_B \qty(\frac{\hbar\omega/2k_BT}{\sinh\hbar\omega/2k_BT})^2\simeq k_B
\end{aligned}\end{equation}
となる。
なので古典的に扱ったときと同じ結果になる。
\(T\to0\)の低温極限では\(\sinh\hbar\omega/2k_BT\simeq \exp[\hbar\omega/2k_BT/2]\)より
\begin{equation}\begin{aligned}[b]
    C \simeq k_B \qty(\frac{\hbar\omega}{k_BT})^2e^{-\hbar\omega/2k_BT}
\end{aligned}\end{equation}
となり低温では温度特性があるとわかる。

\subsection{}
\(x_0 = 0,x_{N+1}=a(N+1)\)とする。
するとハミルトニアンは
\begin{equation}\begin{aligned}[b]
    H = \sum_{n=1}^{N+1}\qty(\frac{m\omega^2}{2}(x_n-x_{n-1}-a)^2+\frac{p_n^2}{2m})
\end{aligned}\end{equation}
\subsubsection*{答案1}
固定端条件の波動解になると期待されるので
質点の位置を質点の平衡位置を基準としたフーリエ級数で表す。
\begin{equation}\begin{aligned}[b]
    x_n(t)-an &= \sum_{l=1}^{N} \tilde{x}_l(t)\exp[i\frac{\pi ln}{N+1}]\\
    p_n(t)-an &= \sum_{l=1}^{N} \tilde{p}_l(t)\exp[i\frac{\pi ln}{N+1}]\\
\end{aligned}\end{equation}
ハミルトニアンを複素振幅で表すと
\begin{equation}\begin{aligned}[b]
    H &= \sum_{n=1}^{N+1}\qty[\frac{1}{2}m\omega^2\abs{x_n-x_{n-1}-a}^2+\frac{1}{2m}p_n^2]\\
    &= \sum_{n=1}^{N+1}\qty[\frac{1}{2}m\omega^2\abs{\sum_{l=1}^{N}\qty[1-\exp(-i\frac{\pi l}{N+1})]\tilde{x}_l(t)\exp[i\frac{\pi ln}{N+1}]}^2
    +\frac{1}{2m}\abs{\sum_{l=-N}^{N}\tilde{p}_l(t)\exp[i\frac{\pi ln}{N+1}]}^2]\\
    &= (N+1)\sum_{l=1}^{N}\qty[\frac{1}{2}m\omega^2\qty(2-2\cos(\frac{l}{N+1}\pi))\abs{\tilde{x}_l(t)}^2
    +\frac{1}{2}m\abs{\tilde{p}_l(t)}^2]
\end{aligned}\end{equation}
となり、各モード\(l\)ごとに分離した調和振動子ポテンシャルとなっている。
ポテンシャルの形より固有振動数は
\begin{equation}\begin{aligned}[b]
    \omega_l^2 &= \omega^2\qty(2-2\cos(\frac{l}{N+1}\pi))\\
    \omega_l &= \omega\sqrt{2-2\cos(\frac{l}{N+1}\pi)}
\end{aligned}\end{equation}

\subsubsection*{答案2}
\(x_0 = 0,x_{N+1}=a(N+1)\)とする。
するとハミルトニアンは
\begin{equation}\begin{aligned}[b]
    H = \sum_{n=1}^{N+1}\qty(\frac{m\omega^2}{2}(x_n-x_{n-1}-a)^2+\frac{p_n^2}{2m})
\end{aligned}\end{equation}
正準方程式より\(n=1,\cdots,N\)の運動方程式は
\begin{equation}\begin{aligned}[b]
    \begin{cases}
        \dv{x_i}{t}&= \pdv{H}{p_i}\\
        \dv{p_i}{t}&=-\pdv{H}{x_i}
    \end{cases}
    \rightarrow
    \begin{cases}
        \dv{x_i}{t} &= \frac{p_i}{m}\\
        \dv{p_i}{t} &= -m\omega^2(-x_{n-1}+2x_{n}-x_{n+1})
    \end{cases}
\end{aligned}\end{equation}
これより
\begin{equation}\begin{aligned}[b]
    \dv[2]{x_i}{t} &= -\omega^2(-x_{n-1}+2x_{n}-x_{n+1})
\end{aligned}\end{equation}
この微分方程式の解として固定端条件を満たす形
\begin{equation}\begin{aligned}[b]
    x_n(t) = Q_l\sin(\frac{l\pi}{N+1} n)e^{-i\omega_l t}+an,\qquad l=1,2,\cdots,N
\end{aligned}\end{equation}
を式に代入して整理すると
\begin{equation}\begin{aligned}[b]
    -\omega_l^2 Q_l\sin(\frac{l\pi}{N+1} n)e^{-i\omega_l t}
    &= -\omega^2\qty[2-2\cos(\frac{l\pi}{N+1})]Q_l\sin(\frac{l\pi}{N+1} n)e^{-i\omega_l t}
\end{aligned}\end{equation}
これより
\begin{equation}\begin{aligned}[b]
    \omega_l &= \omega \sqrt{2-2\cos(\frac{l\pi}{N+1})}
\end{aligned}\end{equation}

\subsection{}
前問よりハミルトニアンをフーリエ変換すると独立な調和振動子になる。
古典的に扱った時には比熱は振動数によらず、
各モード1つあたりが質点1個当たりの自由度に対応する。
なので比熱は設問1で求めたのと同じ
\begin{equation}\begin{aligned}[b]
    C_1 = k_B
\end{aligned}\end{equation}
となる。


\subsection*{感想}
物量が苦しい。
固定端条件でのフォノン分散とか初めてやった気がする。

\end{document}