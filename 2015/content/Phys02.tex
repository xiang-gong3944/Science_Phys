\documentclass[../../sp_2015.tex]{subfiles}

\graphicspath{{./image/}}

\begin{document}
\section{統計力学: 固体の比熱}
\subsection{}
この系の古典的な分配関数は
\begin{equation}\begin{aligned}[b]
    Z &= \frac{1}{(2\pi\hbar)^N}\qty(\int_{-\infty}^{\infty}dp\int_{-\infty}^{\infty}dx\,e^{-\beta p^2/2m}e^{-\beta m\omega(x-a)^2/2})^N\\
    &= \frac{1}{(2\pi\hbar)^N}\qty(\sqrt{\frac{2m\pi}{\beta}}\sqrt{\frac{2\pi}{\beta m\omega^2}})^N
    =\qty(\frac{k_BT}{\hbar\omega})^N
\end{aligned}\end{equation}
よって内部エネルギーは
\begin{equation}\begin{aligned}[b]
    \frac{U}{N}=-\frac{1}{N}\pdv{\beta}\ln Z = k_BT^2\pdv{T}\ln\frac{k_BT}{\hbar\omega} = kB_T
\end{aligned}\end{equation}
比熱は
\begin{equation}\begin{aligned}[b]
    C = \dv{U}{T}=k_B
\end{aligned}\end{equation}
となる。
位置の期待値を求める。
1つの粒子の分配関数を考えることにより\(\ev{x_1}\)は
\begin{equation}\begin{aligned}[b]
    \ev{x_i} &= \frac{\int dp\,e^{-\beta p^2/2m}\int dx\, xe^{-\beta m\omega^2(x-a)^2/2}}{\int dp\,e^{-\beta p^2/2m}\int dx\, e^{-\beta m\omega^2(x-a)^2/2}}\\
    &= a
\end{aligned}\end{equation}
そして\(\ev{x_1^2}\)は
\begin{equation}\begin{aligned}[b]
    \ev{x_i} &= \frac{\int dp\,e^{-\beta p^2/2m}\int dx\, x^2e^{-\beta m\omega^2(x-a)^2/2}}{\int dp\,e^{-\beta p^2/2m}\int dx\, e^{-\beta m\omega^2(x-a)^2/2}}\\
    &= \frac{\int dx\, \qty[(x-a)^2+2a(x-a)+a^2]e^{-\beta m\omega^2(x-a)^2/2}}{\int dx\, e^{-\beta m\omega^2(x-a)^2/2}}\\
    &= \frac{k_BT}{m\omega^2}+a^2
\end{aligned}\end{equation}
となる。よって分散は
\begin{equation}\begin{aligned}[b]
    \ev{x_1^2}-\ev{x_1}^2=\frac{k_BT}{m\omega^2}
\end{aligned}\end{equation}

\subsection{}
1つの質点の分散関係は
\begin{equation}\begin{aligned}[b]
    Z = \sum_{n=0}^{\infty}\bra{n}e^{-\beta H}\ket{n}
    = \sum_{n=0}^{\infty}e^{-\beta\hbar\omega(n+1/2)}
    =\frac{e^{-\beta\hbar\omega/2}}{1-e^{-\beta\hbar\omega}}
    = \frac{1}{2\sinh(\beta\hbar\omega/2)}
\end{aligned}\end{equation}
これより位置の期待値は
\begin{equation}\begin{aligned}[b]
    \ev{x} = \frac{1}{Z}\sum_{n=0}^{\infty}\bra{n}Xe^{-\beta H}\ket{n}
    = \frac{1}{Z}\sum_{n=0}^{\infty}e^{-\beta\hbar\omega(n+1/2)}
        \sqrt{\frac{\hbar}{2m\omega}}\bra{n}\qty(\sqrt{n}\ket{n-1}+\sqrt{n+1}\ket{n+1})
    = 0
\end{aligned}\end{equation}
なので分散は
\begin{equation}\begin{aligned}[b]
    \ev{x^2} &= \frac{1}{Z}\sum_{n=0}^{\infty}\bra{n}Xe^{-\beta H}\ket{n}\\
    &= \frac{1}{Z}\sum_{n=0}^{\infty}e^{-\beta\hbar\omega(n+1/2)}
        \frac{\hbar}{2m\omega}\bra{n}\qty(\sqrt{n(n-1)}\ket{n-2}+n\ket{n}+(n+1)\ket{n}+\sqrt{(n+1)(n+2)}\ket{n+2})\\
    &= \frac{1}{Z}\sum_{n=0}^{\infty} \frac{1}{m\omega^2}\hbar\omega\qty(n+\frac{1}{2})e^{-\beta\hbar(n+1/2)}
    = -\frac{1}{m\omega^2}\pdv{\beta}\ln Z =\frac{\hbar}{2m\omega}\coth\frac{\beta\hbar\omega}{2}
\end{aligned}\end{equation}

また、内部エネルギーは
\begin{equation}\begin{aligned}[b]
    U = -\pdv{\beta}\ln Z
    = \pdv{\beta} \ln\sinh\frac{\beta\hbar\omega}{2}
    =\frac{\hbar\omega}{2}\coth\frac{\beta\hbar\omega}{2}
\end{aligned}\end{equation}
比熱は
\begin{equation}\begin{aligned}[b]
    C = \dv{U}{T}
    = -k_B\beta^2\dv{\beta}\coth\frac{\beta\hbar\omega}{2}
    = k_B\qty(\frac{\hbar\omega}{2k_BT})^2\qty[\frac{\cosh^2\beta\hbar\omega/2}{\sinh^2\beta\hbar\omega/2}-1]
    = k_B \qty(\frac{\hbar\omega/2k_BT}{\sinh\hbar\omega/2k_BT})^2
\end{aligned}\end{equation}
となる。
\(T\to\infty\)の高温極限では\(\sinh\hbar\omega/2k_BT\simeq\hbar\omega/2k_BT\)より
\begin{equation}\begin{aligned}[b]
    C = k_B \qty(\frac{\hbar\omega/2k_BT}{\sinh\hbar\omega/2k_BT})^2\simeq k_B
\end{aligned}\end{equation}
となる。
なので古典的に扱ったときと同じ結果になる。
\(T\to0\)の低温極限では\(\sinh\hbar\omega/2k_BT\simeq \exp[\hbar\omega/2k_BT/2]\)より
\begin{equation}\begin{aligned}[b]
    C \simeq k_B \qty(\frac{\hbar\omega}{k_BT})^2e^{-\hbar\omega/2k_BT}
\end{aligned}\end{equation}
となり低温では温度特性があるとわかる。

\subsection{}

\(x_0 = 0,x_{N+1}=a(N+1)\)とする。
するとハミルトニアンは
\begin{equation}\begin{aligned}[b]
    H = \sum_{n=0}^{N}\qty(\frac{m\omega^2}{2}(x_{n+1}-x_{n}-a)^2+\frac{p_n^2}{2m})
\end{aligned}\end{equation}
となる。
固定端条件の波動方程式のような解となると期待できるので、
位置と運動量を
\begin{equation}\begin{aligned}[b]
    x_n(t) = \sum_{l=1}^{N}Q_l(t)\sin(\frac{l\pi}{N+1} n)+an,\qquad
    p_n(t) = \sum_{l=1}^{N}P_l(t)\sin(\frac{l\pi}{N+1} n)
\end{aligned}\end{equation}
のようにおく。
三角関数の直交性
\begin{equation}\begin{aligned}[b]
    \sum_{n=0}^{N}\sin(\frac{l\pi}{N}n)\sin(\frac{l'\pi}{N}n) = \frac{N}{2}\delta_{l,l'}
\end{aligned}\end{equation}
と合わせるとハミルトニアンは、
\begin{equation}\begin{aligned}[b]
    H &= \sum_{l,l'=1}^{N}\sum_{n=1}^{N}\qty[\frac{P_lP_{l'}}{2m}\sin(\frac{l\pi}{N+1}n)\sin(\frac{l'\pi}{N+1}n)]
    +\sum_{n=1}^{N}\frac{m\omega^2}{2} x_n(-x_{n+1}+2x_n-x_{n-1})\\
    &= \frac{N+1}{2}\sum_{l=1}^{N}\frac{P_l^2}{2m}
    +\sum_{l,l'=1}^{N}\sum_{n=1}^{N}\frac{m\omega^2Q_lQ_{l'}}{2} \qty{2-2\cos(\frac{l'}{N+1}\pi)}\sin(\frac{l\pi}{N+1}n)\sin(\frac{l'\pi}{N+1}n)\\
    &= \frac{N+1}{2}\sum_{l=1}^{N}\qty[\frac{P_l^2}{2m}
     +\frac{mQ_l^2}{2}\qty{\omega\sqrt{2-2\cos(\frac{l}{N+1}\pi)}}]
    \equiv \frac{N+1}{2}\sum_{l=1}^{N}\qty(\frac{P_l^2}{2m}
     +\frac{m\omega_l^2Q_l^2}{2})
\end{aligned}\end{equation}
となる。
途中
\begin{equation}\begin{aligned}[b]
    -x_{n+1}&+2x_n-x_{n-1}\\
    &= \sum_{l'=1}^{N}Q_{l'}\qty[
        -\sin(\frac{l'\pi}{N+1}(n+1))+2\sin(\frac{l'\pi}{N+1}n)
        -\sin(\frac{l'\pi}{N+1}(n-1))]\\
    &= \sum_{l'=1}^{N}Q_{l'}\qty[
        -\sin(\frac{l'\pi}{N+1}n)\cos(\frac{l'\pi}{N+1}n)+2\sin(\frac{l'\pi}{N+1}n)
        -\sin(\frac{l'\pi}{N+1}n)\cos(\frac{l'\pi}{N+1}n)]\\
    &= \sum_{l'=1}^{n}\qty{2-2\cos(\frac{l'}{N+1}\pi)}\sin(\frac{l\pi}{N+1}n)
\end{aligned}\end{equation}
を使った。
このハミルトニアンより、各モードは独立した調和振動子で固有振動数が\(\omega_l\)になっているとわかる。


\subsection{}
前問よりこの系は独立な調和振動子が\(N\)個ある系とみなせる。
古典的に扱った調和振動子系では比熱は振動数によらず、
各モード1つあたりが質点1個当たりの自由度に対応する。
なので比熱は設問1で求めたのと同じ
\begin{equation}\begin{aligned}[b]
    C_1 = k_B
\end{aligned}\end{equation}
となる。

\subsection{}
\subsubsection{答案1}
低温かつ\(N\gg1\)では
固有振動数は
\begin{equation}\begin{aligned}[b]
    \omega_l \simeq \omega \frac{l\pi}{N}
\end{aligned}\end{equation}
と近似できる。
各モードの振動子を量子力学的扱うと設問1と同じ比熱が得られる。
これをすべて足し合わせると比熱が求まる。
よって\(N\)が大きく、低温では
\begin{equation}\begin{aligned}[b]
    C &= \frac{1}{N}\sum_{l=1}^{N} k_B \qty(\frac{\hbar\omega_l/2k_BT}{\sinh\hbar\omega_l/2k_BT})^2
    = k_B\frac{1}{\beta\hbar\omega \pi}\sum_{l=1}^{N}\frac{\beta\hbar\omega\pi}{N}\frac{(\beta\hbar \omega_l)^2e^{\beta\hbar\omega_l}}{(e^{\beta\hbar\omega_l}-1)^2}\\
    &\simeq k_B \frac{k_BT}{\pi\hbar\omega}\int_{0}^{\beta\hbar\omega\pi}\frac{x^2e^x}{(e^x-1)^2}dx
    \simeq k_B \frac{k_BT}{\pi\hbar\omega}\int_{0}^{\infty}\frac{x^2e^x}{(e^x-1)^2}dx
    = \frac{\pi k_B^2}{3\hbar\omega}T
\end{aligned}\end{equation}

\subsubsection{答案2}
波数を\(k=l\pi/a(N+1)\)とすると
分散関係は\(N\gg1\)では波数は連続量とみなせる。
このとき分散関係は
\begin{equation}\begin{aligned}[b]
    \omega(k) = \omega_0\sqrt{2-2\cos(ka)}=2\omega_0\sin(\frac{ka}{2})\simeq a\omega_0 k
\end{aligned}\end{equation}
となる。
ここで、\(\omega_0\)としたのはもとの調和振動子の結合定数、
\(\omega(k)\)は\(\omega_l\)に対応するものである。
これよりフォノンの状態密度\(D(\omega)\)は
\begin{equation}\begin{aligned}[b]
    D(\omega) = \frac{a}{\pi}\dv{k}{(\omega(k))} = \frac{1}{\pi\omega_0}
\end{aligned}\end{equation}
フォノンはボーズ分布に従うので、1つのフォノンの内部エネルギーと比熱は
\begin{equation}\begin{aligned}[b]
    E &= \int_0^\infty \hbar\omega \frac{D(\omega)}{e^{\beta\hbar\omega}-1}d\omega
    = \frac{1}{\pi\beta^2\hbar\omega_0}\int_{0}^{\infty}\frac{\beta\hbar\omega}{e^{\beta\hbar\omega}-1}d(\beta\hbar\omega)
    = \frac{\pi k_B^2}{6\hbar\omega_0}T^2\\
    C &= \dv{E}{T}= \frac{\pi k_B^2}{3\hbar\omega}T
\end{aligned}\end{equation}



\subsection*{感想}
物量が苦しい。
固定端条件でのフォノン分散とか初めてやった気がする。
直接ハミルトニアンをフーリエ変換して、
各モードの和へと具体的に変形してやるというルートが楽なのかしら。

この対角化の際に初見では思いつかないであろう変形をポテンシャルエネルギーにした。
それのお気持ちとしては、
ハミルトニアンから力
\begin{equation*}\begin{aligned}[b]
    F_n=-\pdv{H}{x_n}=-\frac{m\omega_0^2}{2}(-x_{n+1}+2x_n-x_{n-1})
\end{aligned}\end{equation*}
が得られて、
ポテンシャルエネルギーが
\begin{equation*}\begin{aligned}[b]
    \sum_{n=1}^{N}U_n = \sum_{n=1}^{N}(-F_n x_n) = \sum_{n=1}^{N}\frac{m\omega_0^2}{2}x_n(-x_{n+1}+2x_n-x_{n-1})
\end{aligned}\end{equation*}
というにかけるというなイメージで変形した。

\end{document}