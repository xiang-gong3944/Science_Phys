\documentclass[../../sp_2015.tex]{subfiles}

\graphicspath{{./image/}}

\begin{document}
\section{力学: 相対論的力学}
\subsection{}
ロケット本体の運動量変化は
\begin{equation}\begin{aligned}[b]
    \dv{t}m(t)v(t) = m(t)a+\dv{m}{t}v(t)
\end{aligned}\end{equation}
時刻\(t\)の微小時間\(dt\)に放出されたガス\(dm\)が持っている運動量は
\begin{equation}\begin{aligned}[b]
    dm(v(t)-v_{gas}) = \dv{m}{t}(v(t)-v_{gas}) dt
\end{aligned}\end{equation}
とわかる。運動量保存則よりこれらの和が\(0\)であるので
\begin{equation}\begin{aligned}[b]
    m(t)a+\dv{m}{t}v(t) &= \dv{m}{t}(v(t)-v_{gas})\\
    \dv{m}{t} &= -\frac{a}{v_{gas}}m(t)
\end{aligned}\end{equation}

\subsection{}
線形微分方程式より解は
\begin{equation}\begin{aligned}[b]
    m(t) = m_0e^{-at/v_{gas}}
\end{aligned}\end{equation}

\subsection{}
イ:
\begin{equation}\begin{aligned}[b]
     d\tau = cdt
\end{aligned}\end{equation}

ロ:
\begin{equation}\begin{aligned}[b]
    u^\mu\odot u^\mu=-u^0u^0 + u^1u^1 = \frac{-c^2dt^2+dx^2}{d\tau^2} = c^2\frac{ds^2}{-ds^2}= -c^2
\end{aligned}\end{equation}

ハ:
\begin{equation}\begin{aligned}[b]
    \dv{\tau} (u^\mu\odot u^\mu) &= \dv{\tau}(-c^2)\\
        u^\mu\odot a^\mu &= 0
\end{aligned}\end{equation}

ニ・ホ:
\(a^0\)について
\begin{equation}\begin{aligned}[b]
    u^0 &= \dv{ct}{\tau} =\frac{c}{\sqrt{1-v^2/c^2}}\\
    a^0 &= \dv{u^0}{\tau} = \dv{t}{\tau}\dv{u^0}{t}
        = \frac{v/c}{(1-v^2/c^2)^{2}}\dv{v}{t}
\end{aligned}\end{equation}

\(a^1\)について
\begin{equation}\begin{aligned}[b]
    u^1 &= \dv{x^1}{\tau} = \dv{t}{\tau}\dv{x^1}{t} = \frac{v}{\sqrt{1-v^2/c^2}}\\
    a^1 &= \dv{u^1}{\tau} = \dv{t}{\tau}\dv{u^1}{t}
        = \frac{1}{\sqrt{1-v^2/c^2}}\qty(\frac{1}{\sqrt{1-v^2/c^2}}+\frac{v^2/c^2}{(1-v^2/c^2)^{3/2}})\dv{v}{t}
        = \frac{1}{(1-v^2/c^2)^2}\dv{v}{t}
\end{aligned}\end{equation}

これより\(\alpha\)は
\begin{equation}\begin{aligned}[b]
    \alpha^2 &= -a^0a^0+a^1a^1
    = \frac{1-v^2/c^2}{(1-v^2/c^2)^4}\qty(\dv{v}{t})^2
    = \frac{1}{(1-v^2/c^2)^3}\qty(\dv{v}{t})^2\\
    \alpha &= \frac{1}{(1-v^2/c^2)^{3/2}}\dv{v}{t}
\end{aligned}\end{equation}

\(\alpha\)を使って\(\alpha^0,\alpha^1\)を表すと
\begin{equation}\begin{aligned}[b]
    a^0 &= \alpha \frac{u^1}{c}, &
    a^1 &= \alpha \frac{u^0}{c}
\end{aligned}\end{equation}


\subsection*{感想}
相対論的力学とか使わないからもう忘れた。
それでもこれを知ってるよねと問われたので苦しい。

\end{document}