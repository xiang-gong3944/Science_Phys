\documentclass[../../sp_2018.tex]{subfiles}

\graphicspath{{./image/}}

\begin{document}
\setcounter{section}{2}
\section{力学:万有引力}
\subsection{}
EL.eq より\(r\)方向は
\begin{equation}\begin{aligned}[b]
    \pdv{\mathcal{L}}{r}-\dv{t}\pdv{\mathcal{L}}{\dot{r}} &=0\\
    \mu r\dot{\varphi}^2-\frac{G\mu M}{r^2} -\mu\ddot{r} &=0\\
    \mu\ddot{r} &= \mu r\dot{\varphi}^2 - \frac{G\mu M}{r^2}
\end{aligned}\end{equation}
\(\varphi\)方向は
\begin{equation}\begin{aligned}[b]
    \pdv{\mathcal{L}}{\varphi}-\dv{t} \pdv{\mathcal{L}}{\dot{\varphi}} &= 0\\
    \dv{t}\mu r^2\varphi^2 &= 0
\end{aligned}\end{equation}

\subsection{}
\(r\)に共役な運動量\(p\)は
\begin{equation}\begin{aligned}[b]
    p = \pdv{\mathcal{L}}{\dot{r}} = \mu\dot{r}
\end{aligned}\end{equation}
\(\varphi\)に共役な運動量\(J\)は
\begin{equation}\begin{aligned}[b]
    J = \pdv{\mathcal{L}}{\dot{\varphi}} = \mu r^2\dot{\varphi}
\end{aligned}\end{equation}
エネルギーとハミルトニアンは今の場合等しいので
\begin{equation}\begin{aligned}[b]
    E = p\dot{r} + J\dot{\varphi} -\mathcal{L} = \frac{p^2}{2\mu}+\frac{J^2}{2\mu r^2}-\frac{G\mu M}{r}
\end{aligned}\end{equation}

\subsection{}
設問1より角運動量は
\begin{equation}\begin{aligned}[b]
    \dv{J}{t} = \dv{t} \mu r^2\dot{\varphi} = 0
\end{aligned}\end{equation}
となるので、\(J\)は時間によらない運動の定数とわかる。

エネルギーについて、
\begin{equation}\begin{aligned}[b]
    \dv{E}{t} &= \dv{r}{t}\pdv{E}{r}+\dv{p}{t}\pdv{E}{p}\\
    &= \frac{p}{\mu}\qty(-\frac{J^2}{\mu r^3}+\frac{G\mu M}{r^2})+\qty(\frac{J^2}{\mu r^3}-\frac{G\mu M}{r^2})\frac{p}{\mu}\\
    &= 0
\end{aligned}\end{equation}
より、これも時間によらない運動の定数とわかる。

\subsection{}
\begin{equation}\begin{aligned}[b]
    \dv{E}{t} &= -L_{GW}\\
    \frac{G\mu M}{2a^2}\dv{a}{t} &= -\frac{32G^4}{5c^5}\frac{\mu^2M^3}{a^5}\\
    \dv{a}{t} &=-\frac{64G^3}{5c^5}\frac{\mu M^2}{a^3}
\end{aligned}\end{equation}

\subsection{}
初期条件を満たすように与え垂れた一階の微分方程式を解くと
\begin{equation}\begin{aligned}[b]
    \dv{P}{t} &= -A\qty{\frac{P_c}{P}}^{5/3}\\
    \qty(\frac{P}{P_c})^{5/3} \dv{P}{t} &= -A\\
    \frac{3}{8}\qty(\frac{P}{P_c})^{8/3} &= -A\frac{t}{P_c}+\frac{3}{8}\qty(\frac{P_0}{P_c})^{8/3}
\end{aligned}\end{equation}
これより\(P=0\)となる時間は
\begin{equation}\begin{aligned}[b]
    0 &= -A\frac{\tau_{GW}}{P_c}+\frac{3}{8}\qty(\frac{P_0}{P_c})^{8/3}\\
    \tau_{GW} &=\frac{3P_c}{8A}\qty(\frac{P_0}{P_c})^{8/3}
\end{aligned}\end{equation}

\subsection{}
ケプラーの方程式の両辺を時間微分すると
\begin{equation}\begin{aligned}[b]
    GMP\dv{P}{t} &= 12\pi^2 a^2\dv{a}{t}\\
    \dv{P}{t} &= \frac{6\pi^2}{GN}\frac{a^2}{P}\qty(-\frac{64G^3}{c^5}\frac{\mu M^2}{a^3})\\
    &= -\frac{2^6\cdot 3\pi (2\pi)^{5/3}}{5}\qty(\frac{G\mu^{3/5}M^{2/5}c^{-3}}{P})^{5/3}\\
    &\equiv -A \qty(\frac{P_c}{P})^{5/3}
\end{aligned}\end{equation}
より
\begin{equation}\begin{aligned}[b]
    P_c &=G\mu^{3/5}M^{2/5}c^{-3} & A &= \frac{192\pi (2\pi)^{5/3}}{5}
\end{aligned}\end{equation}
とすれは式(6)が得られる。

\subsection{}
\(m=kM_{\odot}\)とすると、\(M=2kM_{\odot},\mu=kM_{\odot}/2\)となり、
\begin{equation}\begin{aligned}[b]
    P_c =G\mu^{3/5}M^{2/5}c^{-3}=\frac{k}{2^{1/5}}GM_\odot c^{-3}=\frac{k}{2.2}\times10^{-5}\,\si{sec}
\end{aligned}\end{equation}
とわかる。
これと設問5の結果を合わせると
\begin{equation}\begin{aligned}[b]
    \tau_{GW} &=\frac{3P_c}{8A}\qty(\frac{P_0}{P_c})^{8/3}\\
    0.15 \,\si{sec} &= \frac{3}{8\times2500}\frac{k}{2.2}\qty(\frac{0.06\,\si{sec}}{k/2.2\si{sec}}\times 10^5)^{8/3}\,\si{sec}
\end{aligned}\end{equation}

\subsection*{感想}


\end{document}
