\documentclass[../../sp_2018.tex]{subfiles}

\graphicspath{{./image/}}

\begin{document}
\setcounter{section}{0}
\section{量子力学: 調和振動子・摂動論}
\subsection{}
交換関係は
\begin{equation}\begin{aligned}[b]
    [a,a^\dagger] &= \frac{m\omega}{2\hbar}\qty[x+\frac{ip}{m\omega},x-\frac{ip}{m\omega}]\\
    &= \frac{m\omega}{2\hbar}\qty[x,-\frac{ip}{m\omega}]+\frac{m\omega}{2h}\qty[\frac{ip}{m\omega},x]\\
    &= \frac{-i}{2\hbar}i\hbar+\frac{i}{2\hbar}(-i\hbar)
    = 1
\end{aligned}\end{equation}
\begin{equation}\begin{aligned}[b]
    [N,a] = [a^\dagger a,a] = [a^\dagger,a]a = -a
\end{aligned}\end{equation}
\begin{equation}\begin{aligned}[b]
    [N,a^\dagger] = [a^\dagger a,a^\dagger] = a^\dagger[a,a^\dagger]a = a^\dagger
\end{aligned}\end{equation}
また、
\begin{equation}\begin{aligned}[b]
    N &= \frac{m\omega}{2\hbar}\qty(x-\frac{ip}{m\omega})\qty(x-\frac{ip}{m\omega})\\
    &=\frac{m\omega}{2\hbar}\qty(x^2+\frac{i}{m\omega}[x,p]+\frac{p^2}{m^2\omega^2})\\
    \hbar\omega &= \frac{1}{2}m\omega^2+\frac{1}{2m}p^2-\frac{1}{2}\hbar\omega = H_0-\frac{1}{2}\hbar\omega\\
    H_0 &= \hbar\omega\qty(N+\frac{1}{2})
\end{aligned}\end{equation}
よって
\begin{equation}\begin{aligned}[b]
    \alpha = 1,\,\beta =\frac{1}{2}
\end{aligned}\end{equation}

\subsection{}
\subsubsection*{(i)}
\(N\)の固有ケットを\(\ket{\phi_0}\), 固有値を\(n_0\)とする。
\begin{equation}\begin{aligned}[b]
    \mel{\psi_0}{N}{\psi_0}&=n\braket{\psi_0}{\psi_0} = n\\
    \mel{\psi_0}{a^\dagger a}{\psi_0} &= \abs{a\ket[\psi_0]}^2 \geq 0
\end{aligned}\end{equation}
より\(N\)の固有値\(n\)は非負である。
\subsubsection*{(ii)}
最低固有値を取るケットを\(\ket{0}\), 固有値を\(n_0\)とする。
\(n_0\neq0\)とすると、
\begin{equation}\begin{aligned}[b]
    Na\ket{0} = (aN+[N,a])\ket{0}=(n_0-1)a\ket{0}
\end{aligned}\end{equation}
となり、\(n_0\)より小さい固有値を作ることができるので、矛盾。
なので、最小の固有値は\(n_0=0\)である。

適当な固有ケット\(\ket{k}\)と固有値\(k\)を考える。
\begin{equation}\begin{aligned}[b]
    Na^\dagger\ket{k} &= (a^\dagger N+[N,a^\dagger])\ket{k} = (k+1)\ket{k}\\
    N\ket{k+1} &= (k+1)\ket{k+1}
\end{aligned}\end{equation}
なので、\(a^\dagger\)は固有値を1つ増やす状態であるのがわかる。
これと最低固有値が\(0\)であることを組み合わせると、
\(N\)の固有値は整数であるとわかる。

\subsection{}
前問より、\(N\)の\(n\)番目の固有値は\(n\)であるので、ハミルトニアンの\(n\)番目のエネルギーは
\begin{equation}\begin{aligned}[b]
    E_n = \hbar\omega\qty(n+\frac{1}{2})
\end{aligned}\end{equation}

前問より、\(\ket{n}\propto (a^\dagger)^n\ket{0}\)になることがわかる。
規格化条件により帰納的に比例定数を求める。
\begin{equation}\begin{aligned}[b]
    1 &= \abs{\ket{n}}^2\\
    &\propto \abs{a^\dagger\ket{n-1}}^2
    =\bra{n-1}aa^\dagger\ket{n-1} = \bra{n-1}(N+1)\ket{n-1} =n
\end{aligned}\end{equation}
これより
\begin{equation}\begin{aligned}[b]
    \ket{n} =\frac{a^\dagger}{n}\ket{n-1} = \frac{(a^\dagger)^n}{\sqrt{n!}}\ket{0}
\end{aligned}\end{equation}

\subsection{}
\subsubsection*{(i)}
摂動を受けた固有ケットを
\begin{equation}\begin{aligned}[b]
    \ket{n}' = \ket{n}+\ket{n^{(1)}}+\ket{n^{(2)}}+\cdots
\end{aligned}\end{equation}
のように右上に\(V\)の次数を書く。

このときの固有値方程式は
\begin{equation}\begin{aligned}[b]
    (H_0+V)\ket{n}' = (E_n^{(0)}+E_n^{(2)}+E_n^{(2)}+\cdots)\ket{n}
\end{aligned}\end{equation}
のように書かれる。
この式について、\(V\)の1次の項を取り出すと
\begin{equation}\begin{aligned}[b]
    H_0\ket{n^{(1)}}+V\ket{n} = E_n^{(0)}\ket{n^{(1)}}+E_n^{(1)}\ket{n}
\end{aligned}\end{equation}
これの左から\(\bra{m}\)を作用させると
\begin{equation}\begin{aligned}[b]
    E_m^{(0)}\braket{m}{n^{(1)}}+V_{mn} = E_n^{(0)}\braket{m}{n^{(1)}}+E_n^{(1)}\delta_{mn}
\end{aligned}\end{equation}
\(m=n\)の場合を考えると
\begin{equation}\begin{aligned}[b]
    E_n^{(1)} = V_{nn}
\end{aligned}\end{equation}
\(m\neq n\)の場合を考えると
\begin{equation}\begin{aligned}[b]
    \braket{m}{n^{(1)}} = -\frac{V_{mn}}{E_m^{(0)}-E_n^{(0)}}
\end{aligned}\end{equation}

固有値方程式の\(V\)の2次の項は
\begin{equation}\begin{aligned}[b]
    H_0\ket{n^{(2)}}+V\ket{n^{(1)}} =E_n^{(0)}\ket{n^{(2)}}+E_n^{(1)}\ket{n^{(1)}}+E_n^{(2)}\ket{n}
\end{aligned}\end{equation}
これに\(\bra{n}\)を左から作用させると、
\begin{equation}\begin{aligned}[b]
    \bra{n}V\ket{n^{(1)}} &=E_n^{(2)}\\
    E_n^{(2)} &= \sum_{l=0}^{\infty}\bra{n}V\ket{l}\braket{l}{n^{(1)}}=-\sum_{l=0}^{\infty}\frac{\abs{V_{lm}}^2}{E_{l}^{(0)}-E_n^{(0)}}
\end{aligned}\end{equation}

\subsubsection*{(ii)}
\begin{equation}\begin{aligned}[b]
    V_{k0}=\mel{k}{\lambda x^4}{0} = \lambda\qty(\frac{\hbar}{2m\omega})^2\mel{k}{(a+a^\dagger)^4}{0}
\end{aligned}\end{equation}
これより、\(k>4\)のときには
\begin{equation}\begin{aligned}[b]
    V_{k0} = 0
\end{aligned}\end{equation}
\(k=4\)のときには、
\begin{equation}\begin{aligned}[b]
    V_{40}&=\lambda\qty(\frac{\hbar}{2m\omega})^2\mel{4}{(a^\dagger)^4}{0}=\sqrt{24}\lambda\qty(\frac{\hbar}{2m\omega})^2
\end{aligned}\end{equation}
\(k=3\)のときには、
\begin{equation}\begin{aligned}[b]
    V_{30}=0
\end{aligned}\end{equation}
\(k=2\)のときには、
\begin{equation}\begin{aligned}[b]
    V_{20}
    &=\lambda\qty(\frac{\hbar}{2m\omega})^2\mel{4}{(aa^\dagger a^\dagger a^\dagger+a^\dagger a a^\dagger a^\dagger+ a^\dagger a^\dagger a a^\dagger)}{0}
    =\qty(3\sqrt{2}+2\sqrt{3}+\sqrt{6})\lambda\qty(\frac{\hbar}{2m\omega})^2
\end{aligned}\end{equation}
\(k=1\)のときには、
\begin{equation}\begin{aligned}[b]
    V_{10}=0
\end{aligned}\end{equation}
\(k=0\)のときには、
\begin{equation}\begin{aligned}[b]
    V_{00}
    &=\lambda\qty(\frac{\hbar}{2m\omega})^2\mel{4}{(aaa^\dagger a^\dagger+aa^\dagger aa^\dagger)}{0}
    =3\lambda\qty(\frac{\hbar}{2m\omega})^2
\end{aligned}\end{equation}

よって
\begin{equation}\begin{aligned}[b]
    E_0 &= E_0^{(0)}+E_0^{(1)}+E_0^{(2)}\\
    &= \frac{1}{2}\hbar\omega + 3\lambda\qty(\frac{\hbar}{2m\omega})^2
        -6(4+\sqrt{6}+\sqrt{3}+\sqrt{2})\frac{\lambda^2}{\hbar\omega}\qty(\frac{\hbar}{2m\omega})^4
\end{aligned}\end{equation}


\subsection*{感想}
教科書みたいな問題で驚いた。


\end{document}
