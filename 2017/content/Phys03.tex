\documentclass[../../sp_2017.tex]{subfiles}

\graphicspath{{./image/}}

\begin{document}
\setcounter{section}{2}
\section{電磁気学: 電磁波の放射}
\subsection{}
スカラーポテンシャルは
\begin{equation}\begin{aligned}[b]
    \phi_0(\vb*{r})
    &=\frac{q}{4\pi\varepsilon_0\sqrt{r^2-rd\cos\theta +d^2/4}}-\frac{q}{4\pi\varepsilon_0\sqrt{r^2+rd\cos\theta +d^2/4}}\\
    &\simeq\frac{q}{4\pi\varepsilon_0 r}\qty(1+\frac{d}{2r}\cos\theta)-\frac{q}{4\pi\varepsilon_0r}\qty(1-\frac{d}{2r}\cos\theta)\\
    &= \frac{qd\cos\theta}{4\pi\varepsilon_0 r^2}
    =\frac{p\cdot r}{4\pi\varepsilon_0 r^3}
\end{aligned}\end{equation}
これより静電場は
\begin{equation}\begin{aligned}[b]
    \vb*{E} &= -\grad \phi_0(\vb*{r}) \\
    &=-\qty(\vb*{e}_r\pdv{r}+\vb*{e}_\theta\frac{\partial}{r\partial\theta}+\vb*{e}_\varphi\frac{\partial}{r\sin\theta\partial\varphi})
    \frac{qd\cos\theta}{4\pi\varepsilon_0 r^2}\\
    &= \frac{qd}{4\pi\varepsilon_0r^3}\qty(2\cos\theta\vb*{e}_r+\sin\theta\vb*{e}_\theta)
\end{aligned}\end{equation}


\subsection{}
\begin{equation}\begin{aligned}[b]
    \vb*{E}(\vb*{r},t) &= -\grad\phi(\vb*{r},t)-\pdv{\vb*{A}(\vb*{r},t)}{t}, & \vb*{B}(\vb*{r},t) &= \curl \vb*{A}(\vb*{r},t)
\end{aligned}\end{equation}


\subsection{}
まず、スカラーポテンシャルについて
\begin{equation}\begin{aligned}[b]
    \div \vb*{E}&=\frac{\rho}{\varepsilon_0}\\
    \div\qty(\grad\phi+\pdv{A}{t}) &=-\frac{\rho}{\varepsilon_0}\\
    \laplacian\phi+\pdv{t}+\div{\vb*{A}} &= -\frac{\rho}{\varepsilon_0}\\
    \laplacian\phi-\frac{1}{c^2}\pdv[2]{\phi}{t} &= -\frac{\rho}{\varepsilon_0}
\end{aligned}\end{equation}
次に、ベクトルポテンシャルについて
\begin{equation}\begin{aligned}[b]
   \curl\vb*{B}-\frac{1}{c^2}\pdv{\vb*{E}}{t} &= \mu_0 \vb*{j}\\
   \curl(\curl\vb*{A})-\frac{1}{c^2}\pdv{t}\qty(-\grad\phi-\pdv{\vb*{A}}{t}) &= \mu_0\vb*{j}\\
   \grad(\div\vb*{A})-\laplacian\vb*{A}+\grad\qty(\frac{1}{c^2}\pdv{\phi}{t})+\frac{1}{c^2}\pdv[2]{\vb*{A}}{t} &= \mu_0 \vb*{j}\\
    \laplacian \vb*{A}-\frac{1}{c^2}\pdv[2]{\phi}{t} &= -\mu_0 \vb*{j}
\end{aligned}\end{equation}

\subsection{}
電荷保存則より
\begin{equation}\begin{aligned}[b]
    q = \int dt\int dS  j_z(\vb*{r},t) dt = \frac{I_0}{i\omega}e^{i\omega t}
\end{aligned}\end{equation}

\subsection{}
電磁波の放射で最も低次の放射は双極子放射で、これは設問1で見たように\(1/r^2\)の依存性になるので、
これより高次の\(1/r^3\)といった項は適宜落としていきながら変形して行く。
よってベクトルポテンシャルは
\begin{equation}\begin{aligned}[b]
    \vb*{A}(\vb*{r},t) &= \frac{\mu_0}{4\pi}\int d^3\vb*{r}'\, \frac{\vb*{j}(\vb*{r}',t-\frac{\abs{\vb*{r}-\vb*{r}'}}{c})}{\abs{\vb*{r}-\vb*{r}'}}\\
    A_z(\vb*{r},t) &= \frac{\mu_0I_0}{4\pi}e^{i\omega t}\int d^3\vb*{r}'\,\frac{e^{-ik\abs{\vb*{r}-\vb*{r}'}}}{\abs{\vb*{r}-\vb*{r}'}}\\
    &\simeq \frac{\mu_0I_0}{4\pi}e^{i\omega t}\int d^3\vb*{r}'\,\frac{e^{-ik\qty(r-r'\cos\theta')}}{r}\qty(1+\frac{r'\cos\theta'}{r})\delta(x)\delta(y)\\
    &\simeq \frac{\mu_0I_0}{4\pi}\frac{e^{i(\omega t-kr)}}{r}\int_{-d/2}^{d/2}dz'\,e^{ikz'}
    = \frac{\mu_0I_0}{2\pi}\frac{e^{i(\omega t-kr)}}{kr}\sin\frac{kd}{2}\\
    &\simeq\frac{\mu_0I_0 d}{4\pi}\frac{e^{i(\omega t-kr)}}{r}\\
    \vb*{A}(\vb*{r},t)
    &= \frac{\mu_0I_0 d}{4\pi}\frac{e^{i(\omega t-kr)}}{r}\qty(\cos\theta \vb*{e}_r-\sin\theta \vb*{e}_\theta)
\end{aligned}\end{equation}
そして、\(r\)の次数を増やさないように磁場を求める。
微分演算子をみると\(\nabla = \vb*{e}_r\partial_r+\vb*{e}_\theta\frac{1}{r}\partial_\theta+\vb*{e}_\varphi\frac{1}{r\sin\theta}\partial_\varphi\)
となっていて、\(r\)方向以外の微分は\(r\)の次数を増やすことがわかる。
なので\(r\)方向の微分だけに注目するの十分である。
\begin{equation}\begin{aligned}[b]
    \vb*{B}(\vb*{r},t) &= \curl \vb*{A}(\vb*{r},t)\\
    &\simeq \pdv{r}\qty[\frac{\mu_0I_0 d}{4\pi}\frac{e^{i(\omega t-kr)}}{r}]\vb*{e}_r\times\qty(\cos\theta \vb*{e}_r-\sin\theta \vb*{e}_\theta)\\
    &\simeq -i\frac{\mu_0I_0 kd}{4\pi}\frac{e^{i(\omega t-kr)}}{r}\sin\theta \vb*{e}_\varphi
    = -i\frac{\mu_0I_0 d\omega}{4\pi c}\frac{e^{i(\omega t-kr)}}{r}\sin\theta \vb*{e}_\varphi
\end{aligned}\end{equation}
となる。2つ目の近似は考えているスケールでは分母の\(r\)はほぼ一定とみなして振動成分の微分だけ見るというものに対応する。
次に電場は、\(r\)付近に電流源はないので
\begin{equation}\begin{aligned}[b]
    \frac{1}{c^2}\pdv{\vb*{E}(\vb*{r},t)}{t}
    &=\curl \vb*{B}(\vb*{r},t)\\
    i\omega\vb*{E}(\vb*{r},t)
    &\simeq \vb*{e}_r\times\vb*{e}_\varphi \pdv{r}\qty[-i\frac{\mu_0I_0 d\omega}{4\pi c}\frac{e^{i(\omega t-kr)}}{r}\sin\theta]\\
    \vb*{E}(\vb*{r},t)
    &\simeq -i\frac{\mu_0I_0 d\omega}{4\pi}\frac{e^{i(\omega t-kr)}}{r}\sin\theta \vb*{e}_\theta
\end{aligned}\end{equation}


\subsection{}
問題の指示に従わず、複素表示でポインティングベクトルを計算する。
そのままだと実表示した電磁場の時間平均と同じ値にならず、
同じ値にするには少し定義が変わって以下のように計算する。
\begin{equation}\begin{aligned}[b]
    \vb*{S} = \frac{1}{2\mu_0}\vb*{E}^{*}\times \vb*{B}
    = \frac{\mu_0}{2c}\qty(\frac{I_0 d\omega}{4\pi})^2\frac{\sin^2\theta}{r^2} \vb*{e}_r
\end{aligned}\end{equation}

\subsection{}
波源との距離は
\begin{equation}\begin{aligned}[b]
    \begin{cases}
        R_1^2 &= r^2 +\frac{D^2}{4}-rD\cos(\frac{\pi}{2}+\varphi)\\
        R_2^2 &= r^2 +\frac{D^2}{4}-rD\cos(\frac{\pi}{2}-\varphi)
    \end{cases} \quad\to\quad
    \begin{cases}
        R_1 &\simeq r \sqrt{1+\frac{D}{r}\sin\varphi}\\
        R_2 &\simeq r \sqrt{1-\frac{D}{r}\sin\varphi}
    \end{cases}\quad\to\quad
    \begin{cases}
        R_1 &\simeq r + \frac{D}{2}\sin\varphi\\
        R_2 &\simeq r - \frac{D}{2}\sin\varphi
    \end{cases}
\end{aligned}\end{equation}
である。
なので電場は
\begin{equation}\begin{aligned}[b]
    \vb*{E} &= \vb*{E}_1+\vb*{E}_2\\
    &= \vb*{e}_zE_0\qty[\cos(\omega t-kR_1-\delta_1)+\cos(\omega t-kR_2-\delta_2)]\\
    &= \vb*{e}_z 2E_0\cos(\frac{k(R_1-R_2)+\delta_1-\delta_2}{2})\cos(\omega t-\frac{k(R_1+R_2)+\delta_1+\delta_2}{2})\\
    &= \vb*{e}_z 2E_0\cos(\frac{kD\sin\varphi+\delta_1-\delta_2}{2})\cos(\omega t-kr-\frac{\delta_1+\delta_2}{2})\\
\end{aligned}\end{equation}
となる。

\(\delta_n=0\)のときには
\begin{equation}\begin{aligned}[b]
    \vb*{E} &=  \vb*{e}_z 2E_0\cos(\frac{kD\sin\varphi}{2})\cos(\omega t-kr)\\
    I &= 4\cos^2\qty(\frac{kD\sin\varphi}{2})\cos^2(\omega t-kr)
\end{aligned}\end{equation}
となる。
強度が最大値となる角度は
\begin{equation}\begin{aligned}[b]
    \frac{kD\sin\varphi}{2} &= n\pi\\
    \sin\varphi &= \frac{2\pi}{kD}n
\end{aligned}\end{equation}
である。
問題文の書き方だとわかりにくいが、
波動の位相は\(kr\)のスケールで変化するので、
電磁波の波長は\(r\)程度である。よって
\(r\gg D\)というのは\(kD\sim D/r \ll 1\)というのを表している。
なのでここで現れた整数は\(n=0\)しかとることができない。
つまり強度が最大になる角度は\(\varphi=0\)である。

これを中心として強度が最大値の半分になる角度は
\begin{equation}\begin{aligned}[b]
    \frac{kD\sin\varphi}{2} &=  \pm\frac{\pi}{4}\\
    \sin\varphi &= \pm \frac{\pi}{2kD}
\end{aligned}\end{equation}
とわかる。よってその角度幅は
\begin{equation}\begin{aligned}[b]
    \delta\varphi = \arcsin(\frac{\pi}{2kD})-\arcsin(-\frac{\pi}{2kD}) = 2\arcsin(\frac{\pi}{2kD})
\end{aligned}\end{equation}
である。
この値を小さくするには\(kD\)を小さくすればよい。よって\(D\)を小さくすればよい。

次に電場の強度が最大になる方向が\(\varphi=\varphi_0\)となるには
\begin{equation}\begin{aligned}[b]
    0 &= \frac{kD\sin\varphi_0+\delta_1-\delta_2}{2}\\
    \delta_2-\delta_1 &= kD\sin\varphi_0
\end{aligned}\end{equation}

\subsection*{感想}
砂川理論電磁気とかジャクソン電磁気を読んだことあるかを聞かれてるのかと思った。
この本だと計算がただただ面倒という印象しかなかったが、
試験問題に収まるように設定を見直すと思ったより単純な計算で双極子放射を求められたので、
勉強になった。

設問5の近似は、双極子放射なので影響は何もかも\(1/r\)に依存して、
\(1/r^2\)より高次の項は4極子以上の放射であるので無視するというのが肝。

\end{document}
